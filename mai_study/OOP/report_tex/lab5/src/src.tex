\section{Описание}

Для доступа к элементам некоторого множества элементов используют специальные объекты, называемые итераторами. В контейнерных типах stl они доступны через методы класса (например, begin() в шаблоне класса vector). Функциональные возможности указателей и итераторов близки, так что обычный указатель тоже может использоваться как итератор.\\

Категории итераторов:
\begin{itemize}
\item{Итератор ввода (input iterator) -- используется потоками ввода.}
\item{Итератор вывода (output iterator) -- используется потоками вывода.}
\item{Однонаправленный итератор (forward iterator) -- для прохода по элементам в одном направлении.}
\item{Двунаправленный итератор (bidirectional iterator) -- способен пройти по элементам в любом направлении. Такие итераторы реализованы в некоторых контейнерных типах stl (list, set, multiset, map, multimap).}
\item{Итераторы произвольного доступа (random access) -- через них можно иметь доступ к любому элементу. Такие итераторы реализованы в некоторых контейнерных типах stl (vector, deque, string, array).}
\end{itemize}


\section{Исходный код}

Описание классов фигур и класса-контейнера остается неизменным.
\begin{lstlisting}[language=C]

template <class N, class T>
class TIterator
{
public:
    TIterator(std::shared_ptr<N> n) {
        cur = n;
    }

    std::shared_ptr<T> operator* () {
        return cur->GetFigure();
    }

    std::shared_ptr<T> operator-> () {
        return cur->GetFigure();
    }

    void operator++() {
        cur = cur->GetNext();
    }

    TIterator operator++ (int) {
        TIterator cur(*this);
        ++(*this);
        return cur;
    }

    void operator--() {
        cur = cur->GetPrev();
    }

    TIterator operator-- (int) {
        TIterator cur(*this);
        --(*this);
        return cur;
    }

    bool operator== (const TIterator &i) {
        return (cur == i.cur);
    }

    bool operator!= (const TIterator &i) {
        return (cur != i.cur);
    }

private:
    std::shared_ptr<N> cur;
};

\end{lstlisting}


\section{Консоль}
\begin{alltt}
karma@karma:~/mai_study/OOP/lab5$ ./run
Choose an operation:
1) Add trapeze
2) Add rhomb
3) Add pentagon
4) Delete figure from list
5) Print list
6) Print list with iterator
0) Exit
2
Enter side: 10
Enter smaller angle: 10
Enter index = 0
Choose an operation:
1) Add trapeze
2) Add rhomb
3) Add pentagon
4) Delete figure from list
5) Print list
6) Print list with iterator
0) Exit
2
Enter side: 9
Enter smaller angle: 20
Enter index = 0
Choose an operation:
1) Add trapeze
2) Add rhomb
3) Add pentagon
4) Delete figure from list
5) Print list
6) Print list with iterator
0) Exit
1
Enter bigger base: 9
Enter smaller base: 8
Enter left side: 7
Enter right side: 6
Enter index = 1
Choose an operation:
1) Add trapeze
2) Add rhomb
3) Add pentagon
4) Delete figure from list
5) Print list
6) Print list with iterator
0) Exit
3
Enter side: 5
Enter index = 0
Choose an operation:
1) Add trapeze
2) Add rhomb
3) Add pentagon
4) Delete figure from list
5) Print list
6) Print list with iterator
0) Exit
5
idx: 0   Sides =  5, type: pentagon

idx: 1   Side = 9, smaller_angle = 20, type: rhomb

idx: 2   Side = 10, smaller_angle = 10, type: rhomb

idx: 3   Smaller base = 8, bigger base = 9, left side = 7, right side = 6, type: trapeze


Choose an operation:
1) Add trapeze
2) Add rhomb
3) Add pentagon
4) Delete figure from list
5) Print list
6) Print list with iterator
0) Exit
4
Enter index = 2
Choose an operation:
1) Add trapeze
2) Add rhomb
3) Add pentagon
4) Delete figure from list
5) Print list
6) Print list with iterator
0) Exit
6
Sides =  5, type: pentagon
Side = 9, smaller_angle = 20, type: rhomb
Smaller base = 8, bigger base = 9, left side = 7, right side = 6, type: trapeze
Choose an operation:
1) Add trapeze
2) Add rhomb
3) Add pentagon
4) Delete figure from list
5) Print list
6) Print list with iterator
0) Exit
0

\end{alltt}

