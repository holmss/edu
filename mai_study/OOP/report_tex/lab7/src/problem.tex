\CWHeader{Лабораторная работа \textnumero 7}

\CWProblem{
Необходимо реализовать динамическую структуру данных -- "Хранилище объектов" и алгоритм работы с ней. "Хранилище объектов" представляет собой контейнер стек. Каждым элементом контейнера является динамическая структура список. Таким образом, у нас получается контейнер в контейнере. Элементов второго контейнера является объект-фигура, определенная вариантом задания. \\
При этом должно выполняться правило, что количество объектов в контейнере второго уровня не больше 5. Т.е. если нужно хранить больше 5 объектов, то создается еще один контейнер второго уровня. \\
Объекты в контейнерах второго уровня должны быть отсортированы по возрастанию площади объекта. При удалении объектов должно выполянться правило, что контейнер второго уровня не должен быть пустым. Т.е. если он становится пустым, то он должен удалиться.


{\bfseries Фигуры:} трапеция, ромб, пятиугольник. \\
{\bfseries Контейнер 1-ого уровня:} связный список. \\
{\bfseries Контейнер 2-ого уровня:} стек.

}
\pagebreak