\section{Описание}

Принцип открытости/закрытости (ОСР) -- принцип ООП, устанавливающий следующее положение: "программные сущности (классы, модули, функции и т.п.) должны быть открыты для расширения, но закрыты для изменения". \\
Контейнер в программировании -- структура (АТД), позволяющая инкапсулировать в себе объекты любого типа. Объектами (переменными) контейнеров являются коллекции, которые уже могут содержать в себе объекты определенного типа. \\
Например, в языке С++, std::list (шаблонный класс) является контейнером, а объект его класса-конкретизации, как например, std::list<int> mylist является коллекцией. \\
Среди "широких масс" программистов наиболее известны контейнеры, построенные на основе шаблонов, однако, существуют и реализации в виде библиотек (наиболее широко известна библиотека GLib). Кроме того, применяются и узкоспециализированные решения. Примерами контейнеров в С++ являются контейнеры из стандартной библиотеки (STL) -- map, vector и т.д. В контейнерах часто встречаются реализации алгоритмов для них. В ряде языков программирования (особенно в скриптовых типа Perl или PHP) контейнеры и работа с ними встроена в язык. \\
Стек -- тип или структура данных в виде набора элементов, которые расположены по принципу LIFO, т.е. "последний пришел, первый вышел". Доступ к элементам осуществляет через обращение к головному элементу (тот, который был добавлен последним). \\

\section{Исходный код}

Описание классов фигур и класса-контейнера списка остается неизменным.\\

\begin{longtable}{|p{7.5cm}|p{7.5cm}|}
\rowcolor{lightgray}
\multicolumn{2}{|c|} {TStack.hpp}\\
\hline
TStack();&Конструктор класса\\
\hline
void Push(const O\&);&Добавление элемента в стек\\
\hline
void Print();&Печать стека\\
\hline
void RemoveByType(const int\&);&Удаление из стека элементов одного типа\\
\hline
void RemoveLesser(const double\&);&Удаление из стека элементов, площадь которых меньше, чем заданная\\
\hline
virtual \textasciitilde TStack();&Деконструктор класса\\
\hline
\end{longtable}

\begin{lstlisting}[language=C]

template <typename Q, typename O> class TStack
{
private:
    class Node {
    public:
        Q data;
        std::shared_ptr<Node> next;
        Node();
        Node(const O&);
        int itemsInNode;
    };

    std::shared_ptr<Node> head;
    int count;
public:
    TStack();

    void Push(const O&);
    void Print();
    void RemoveByType(const int&);
    void RemoveLesser(const double&);

    virtual ~TStack();
};

template <typename T> class TList
{
private:
    class TNode {
    public:
        TNode();
        TNode(const std::shared_ptr<T>&);
        auto GetNext() const;
        auto GetItem() const;
        std::shared_ptr<T> item;
        std::shared_ptr<TNode> next;

        void* operator new(size_t);
        void operator delete(void*);
        static TAllocator nodeAllocator;
    };

    template <typename N, typename M>
    class TIterator {
    private:
        N nodePtr;
    public:
        TIterator(const N&);
        std::shared_ptr<M> operator* ();
        std::shared_ptr<M> operator-> ();
        void operator ++ ();
        bool operator == (const TIterator&);
        bool operator != (const TIterator&);
    };

    int length;

    std::shared_ptr<TNode> head;
    auto psort(std::shared_ptr<TNode>&);
    auto pparsort(std::shared_ptr<TNode>& head);
    auto partition(std::shared_ptr<TNode>&);

public:
    TList();
    virtual ~TList();
    bool PushFront(const std::shared_ptr<T>&);
    bool Push(const std::shared_ptr<T>&, const int);
    bool PopFront();
    bool Pop(const int);
    bool IsEmpty() const;
    int GetLength() const;
    auto& getHead();
    auto&& getTail();
    void sort();
    void parSort();

    TIterator<std::shared_ptr<TNode>, T> begin() {return TIterator<std::shared_ptr<TNode>, T>(head->next);};
    TIterator<std::shared_ptr<TNode>, T> end() {return TIterator<std::shared_ptr<TNode>, T>(nullptr);};

    template <typename A> friend std::ostream& operator<< (std::ostream&, TList<A>&);
};

\end{lstlisting}


\section{Консоль}
\begin{alltt}
karma@karma:~/mai_study/OOP/lab7$ ./run
Choose an operation:
1) Add trapeze
2) Add rhomb
3) Add pentagon
4) Delete by criteria
5) Print
0) Exit
1
Enter bigger base: 2
Enter smaller base: 1
Enter left side: 1
Enter right side: 1
Item was added
Choose an operation:
1) Add trapeze
2) Add rhomb
3) Add pentagon
4) Delete by criteria
5) Print
0) Exit
2
Enter side: 3
Enter smaller angle: 10
Item was added
Choose an operation:
1) Add trapeze
2) Add rhomb
3) Add pentagon
4) Delete by criteria
5) Print
0) Exit
3
Enter side: 3
Item was added
Choose an operation:
1) Add trapeze
2) Add rhomb
3) Add pentagon
4) Delete by criteria
5) Print
0) Exit
3
Enter side: 4
Item was added
Choose an operation:
1) Add trapeze
2) Add rhomb
3) Add pentagon
4) Delete by criteria
5) Print
0) Exit
3
Enter side: 5
Item was added
Choose an operation:
1) Add trapeze
2) Add rhomb
3) Add pentagon
4) Delete by criteria
5) Print
0) Exit
5
Side = 3, smaller_angle = 10, square = 1.56283, type: rhomb
Smaller base = 1, bigger base = 2, left side = 1, right side = 1, square = 1.86603, type: trapeze
Sides =  3, square = 15.4843, type: pentagon
Sides =  4, square = 27.5276, type: pentagon
Sides =  5, square = 43.0119, type: pentagon

Choose an operation:
1) Add trapeze
2) Add rhomb
3) Add pentagon
4) Delete by criteria
5) Print
0) Exit
1
Enter bigger base: 4
Enter smaller base: 3
Enter left side: 3
Enter right side: 3
Item was added
Choose an operation:
1) Add trapeze
2) Add rhomb
3) Add pentagon
4) Delete by criteria
5) Print
0) Exit
2
Enter side: 4
Enter smaller angle: 90
Item was added
Choose an operation:
1) Add trapeze
2) Add rhomb
3) Add pentagon
4) Delete by criteria
5) Print
0) Exit
3
Enter side: 8
Item was added
Choose an operation:
1) Add trapeze
2) Add rhomb
3) Add pentagon
4) Delete by criteria
5) Print
0) Exit
5
Smaller base = 3, bigger base = 4, left side = 3, right side = 3, square = 5.95804, type: trapeze
Side = 4, smaller_angle = 90, square = 16, type: rhomb
Sides =  8, square = 110.111, type: pentagon

Side = 3, smaller_angle = 10, square = 1.56283, type: rhomb
Smaller base = 1, bigger base = 2, left side = 1, right side = 1, square = 1.86603, type: trapeze
Sides =  3, square = 15.4843, type: pentagon
Sides =  4, square = 27.5276, type: pentagon
Sides =  5, square = 43.0119, type: pentagon

Choose an operation:
1) Add trapeze
2) Add rhomb
3) Add pentagon
4) Delete by criteria
5) Print
0) Exit
4
Enter criteria
1) by type
2) lesser than square
1
1) trapeze
2) rhomb
3) pentagon
Enter type
1
Choose an operation:
1) Add trapeze
2) Add rhomb
3) Add pentagon
4) Delete by criteria
5) Print
0) Exit
5
Side = 4, smaller_angle = 90, square = 16, type: rhomb
Sides =  8, square = 110.111, type: pentagon

Side = 3, smaller_angle = 10, square = 1.56283, type: rhomb
Sides =  3, square = 15.4843, type: pentagon
Sides =  4, square = 27.5276, type: pentagon
Sides =  5, square = 43.0119, type: pentagon

Choose an operation:
1) Add trapeze
2) Add rhomb
3) Add pentagon
4) Delete by criteria
5) Print
0) Exit
4
Enter criteria
1) by type
2) lesser than square
2
Enter square
20
Choose an operation:
1) Add trapeze
2) Add rhomb
3) Add pentagon
4) Delete by criteria
5) Print
0) Exit
5
Sides =  8, square = 110.111, type: pentagon

Sides =  4, square = 27.5276, type: pentagon
Sides =  5, square = 43.0119, type: pentagon

Choose an operation:
1) Add trapeze
2) Add rhomb
3) Add pentagon
4) Delete by criteria
5) Print
0) Exit
0

\end{alltt}

