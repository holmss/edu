\section{Описание}

Умный указатель -- класс (обычно шаблонный), имитирующий интерфейс обычного указателя и добавляющий некую новую функциональность, например, проверку границ при доступе или очистку памяти. \\
Существует 3 вида умных указателей стандартной библиотеки С++:
\begin{itemize}
\item{unique ptr -- обеспечивает, чтобы у базового указателя был только один владелец. Может быть передан новому владельцу, но не может быть скопирован или сделан общим. Заменяет auto ptr, использовать который не рекомендуется.}
\item{shared ptr -- умный указатель с подсчитанными ссылками. Используется, когда необходимо присвоить один необработанный указатель нескольким владельцам, например, когда копия указателя возвращается из контейнера, но требуется сохранить оригинал. Необработанный указатель не будет удален до тех пор, пока все владельцы shared ptr не выйдут из области или не откажутся от владения.}
\item{weak ptr -- умный указатель для особых случаев использования с shared ptr. weak ptr предоставляет доступ к объекту, который принадлежит одному или нескольким экземплярам shared ptr, но не участвует в подсчете ссылок. Используется, когда требуется отслеживать объект, но не требуется, чтобы он оставался в активном состоянии.}
\end{itemize}


\section{Исходный код}

\begin{longtable}{|p{7.5cm}|p{7.5cm}|}
\rowcolor{lightgray}
\multicolumn{2}{|c|} {TListItem.cpp}\\
\hline
TListItem(const std::sharedptr<Figure> \&obj);&Конструктор класса\\
\hline
std::sharedptr<Figure> GetFigure() const;&Получение фигуры из узла\\
\hline
std::sharedptr<TListItem> GetNext();&Получение ссылки на следующий узел\\
\hline
std::sharedptr<TListItem> GetPrev();&Получение ссылки на предыдущий узел\\
\hline
void SetNext(std::sharedptr<TListItem> item);&Установка ссылки на следующий узел\\
\hline
void SetPrev(std::sharedptr<TListItem> item);&Установка ссылки на предыдущий узел\\
\hline
friend std::ostream\& operator<<(std::ostream \&os, const TListItem \&obj);&Переопределенный оператор вывода в поток std::ostream\\
\hline
virtual \textasciitilde TListItem(){};&Деконструктор класса\\
\hline
\rowcolor{lightgray}
\multicolumn{2}{|c|} {TList.cpp}\\
\hline
TList();&Конструктор класса\\
\hline
void Push(std::sharedptr<Figure> \&obj);&Добавление фигуры в список\\
\hline
std::sharedptr<Figure> Pop();&Получение фигуры из списка\\
\hline
const bool IsEmpty() const;&Проверка, пуст ли список\\
\hline
uint32t GetLength();&Получение длины списка\\
\hline
friend std::ostream\& operator<<(std::ostream \&os, const TList \&list);&Переопределенный оператор вывода в поток std::ostream\\
\hline
virtual \textasciitilde TList();&Деконструктор класса\\
\hline
\end{longtable}


\begin{lstlisting}[language=C]

class TList
{
public:
    TList();
    void Push(std::shared_ptr<Figure> &obj);
    const bool IsEmpty() const;
    uint32_t GetLength();
    std::shared_ptr<Figure> Pop();
    friend std::ostream& operator<<(std::ostream &os, const TList &list);
    virtual ~TList();

private:
    uint32_t length;
    std::shared_ptr<TListItem> head;

    void PushFirst(std::shared_ptr<Figure> &obj);
    void PushLast(std::shared_ptr<Figure> &obj);
    void PushAtIndex(std::shared_ptr<Figure> &obj, int32_t ind);
    std::shared_ptr<Figure> PopFirst();
    std::shared_ptr<Figure> PopLast();
    std::shared_ptr<Figure> PopAtIndex(int32_t ind);
};

class TListItem
{
public:
    TListItem(const std::shared_ptr<Figure> &obj);

    std::shared_ptr<Figure> GetFigure() const;
    std::shared_ptr<TListItem> GetNext();
    std::shared_ptr<TListItem> GetPrev();
    void SetNext(std::shared_ptr<TListItem> item);
    void SetPrev(std::shared_ptr<TListItem> item);
    friend std::ostream& operator<<(std::ostream &os, const TListItem &obj);

    virtual ~TListItem(){};

private:
    std::shared_ptr<Figure> item;
    std::shared_ptr<TListItem> next;
    std::shared_ptr<TListItem> prev;
};

\end{lstlisting}


\section{Консоль}
\begin{alltt}
karma@karma:~/mai_study/OOP/lab3$ ./run
Choose an operation:
1) Add trapeze
2) Add rhomb
3) Add pentagon
4) Delete figure from list
5) Print list
0) Exit
1
Enter bigger base: 5
Enter smaller base: 2
Enter left side: 2
Enter right side: 2
Enter index = 0
Choose an operation:
1) Add trapeze
2) Add rhomb
3) Add pentagon
4) Delete figure from list
5) Print list
0) Exit
3
Enter side: 5
Enter index = 0
Choose an operation:
1) Add trapeze
2) Add rhomb
3) Add pentagon
4) Delete figure from list
5) Print list
0) Exit
3
Enter side: 10
Enter index = 1
Choose an operation:
1) Add trapeze
2) Add rhomb
3) Add pentagon
4) Delete figure from list
5) Print list
0) Exit
2
Enter side: 6
Enter smaller angle: 40
Enter index = 1
Choose an operation:
1) Add trapeze
2) Add rhomb
3) Add pentagon
4) Delete figure from list
5) Print list
0) Exit
5
idx: 0   Sides =  5, type: pentagon

idx: 1   Side = 6, smaller_angle = 40, type: rhomb

idx: 2   Smaller base = 2, bigger base = 5, left side = 2, right side = 2, type: trapeze

idx: 3   Sides =  10, type: pentagon


Choose an operation:
1) Add trapeze
2) Add rhomb
3) Add pentagon
4) Delete figure from list
5) Print list
0) Exit
4
Enter index = 2
Choose an operation:
1) Add trapeze
2) Add rhomb
3) Add pentagon
4) Delete figure from list
5) Print list
0) Exit
4
Enter index = 0
Choose an operation:
1) Add trapeze
2) Add rhomb
3) Add pentagon
4) Delete figure from list
5) Print list
0) Exit
5
idx: 0   Side = 6, smaller_angle = 40, type: rhomb

idx: 1   Sides =  10, type: pentagon


Choose an operation:
1) Add trapeze
2) Add rhomb
3) Add pentagon
4) Delete figure from list
5) Print list
0) Exit
0
\end{alltt}

