\section{Выводы}

В этом семестре я познакомилась с новой объектно-ориентированной парадигмой программирования. Это методология программирования, основанная на представлении программы в виде совокупности объектов, каждый из которых является экземпляром определенного класса, а классы образуют иерархию наследования. Курс представлял собой полноценный проект, который мы создавали на протяжении семестра. Каждая лабораторная работа была связана с предыдущей, что позволяло находить и исправлять недочеты прошлых версий. Эта практика очень полезна, так оно позволяет нам учиться работать ``на длинных дистанциях``. Мною было сделано 9 лабораторных работ:
\begin{enumerate}
\item{В первой работе я познакомилась с базовыми понятиями ООП, такими как наследование, полиморфизм и инкапсуляция. Было спроектированы классы фигур, заданные вариантом, в которых использовались перегруженные операторы, дружественные функции и операции ввода-вывода из стандартных библиотек.}
\item{Во второй работе я спроектировала динамическую структуру данных -- список. Объекты передаются в методы контейнера по значению.}
\item{В третьей работе я познакомилась с технологией умных указателей. контейнер был переписан с использованием shared ptr. }
\item{В четвертой работе я узнала о шаблонах классов и их проектировании. Был построен шаблон динамической структуры данных. }
\item{В пятой работе у моего контейнера появились итераторы, что облегчило перемещение по списку.}
\item{В шестой работе был создан аллокатор памяти, помогающий оптимизировать выделение и освобождение памяти. Свободные блоки хранятся в аллокаторе в контейнере второго уровня -- стеке. }
\item{В седьмой работе был запрограммирован контейнер в контейнере. То есть в контейнере первого уровня хранятся контейнеры второго уровня, внутри которых хранятся фигуры. Фигуры внутри контейнера второго уровня отсортированы по возрастанию площади. Если в ходе выполнения программы контейнер второго уровня полностью освобождается от фигур, то он удаляется из контейнера первого уровня.}
\item{В восьмой работе я познакомилась с параллельным программированием, что позволило осуществлять быструю сортировку несколько иным образом нежели в классическом варианте: все рекурсивные вызовы теперь выполняются каждый в своем потоке.}
\item{В девятой работе я узнала о лямбда-выражениях. Теперь действия над контейнером первого уровня генерируются в виде команд, которые помещаются в контейнер второго уровня. В рамках данной работы это было еще полезно и как тестирование: можно было убедиться, что контейнер первого уровня работает корректно. }
\end{enumerate}

В итоге выполнения этого проекта я получила хорошие навыки программирования и проектирования на С++. Освоила ряд возможностей языка, которые уже к данному моменту понадобились мне при выполнении работ в других курсах. Думаю, что изучение этой парадигмы совершенно точно необходимо каждому современному программисту вне зависимости от того, чем он занимается. Поэтому нельзя останавливаться на достигнутом и нужно продолжить изучение ООП и С++ в частности.
\pagebreak