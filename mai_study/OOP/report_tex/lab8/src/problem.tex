\CWHeader{Лабораторная работа \textnumero 8}

\CWProblem{
Используя структуры данных, разработанные для лабораторной работы №6 (контейнер 1-ого уровня и классы-фигуры) разработать алгоритм быстрой сортировки для класс-контейнера. \\

Необходимо разработать два вида алгоритма: \\
1. Обычный, без параллельных вызовов. \\
2. С использованием параллельных вызовов. В этом случае, каждый рекурсивный вызов сортировки должен создаваться в отдельном потоке. \\

Для создания потоков ипользовать механизмы:
\begin{itemize}
\item{future}
\item{packaged task/async}
\end{itemize}
\\
Для обеспечения потокобезопасности структур использовать механизмы:
\begin{itemize}
\item{mutex}
\item{lock quard}
\end{itemize}
\\
Программа должна позволять:
\begin{itemize}
\item{Вводить произвольное количество фигур и добавлять их в контейнер.}
\item{Распечатывать содержимое контейнера.}
\item{Удалять фигуры из контейнера.}
\item{Проводить сортировку контейнера.}
\end{itemize}


{\bfseries Фигуры:} трапеция, ромб, пятиугольник. \\
{\bfseries Контейнер:} связный список. \\


}
\pagebreak