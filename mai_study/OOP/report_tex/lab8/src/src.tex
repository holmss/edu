\section{Описание}

Параллельное программирование -- это техника программирования, которая использует преимущества многоядерных или многопроцесорных компьютеров и является подмножеством более широкого понятия многопоточности (multithreading). \\
Параллельное программирование может быть сложным, но его легче понять, если считать его не ``трудным`` , а просто ``немного иным``. Оно включает в себя все черты более традиционного, последовательного программирования, но в параллельном прогаммировании имеются три дополнительных, четко определенных этапа:
\begin{itemize}
\item{Определение параллелизма: анализ задачи с целью выделить подзадачи, которые могут выполняться одновременно.}
\item{Выявление параллелизма: изменение структуры задачи таким образом, чтобы можно было эффективно выполнять подзадачи. Для этого часто требуется найти зависимости между подзадачами и организовать исходный код так, чтобы ими можно было эффективно управлять.}
\item{Выражение параллелизма: реализация параллельного алгоритма в исходном коде с помощью системы обозначений параллельного программирования.}
\end{itemize}

\section{Исходный код}

Описание классов фигур и методов класса-контейнера, определенных раннее, остается неизменным.\\

\begin{lstlisting}[language=C]

template <class T>
std::shared_ptr<TListItem<T>> TList<T>::PSort(std::shared_ptr<TListItem<T>> &head)
{
    if (head == nullptr || head->GetNext() == nullptr) {
        return head;
    }

    std::shared_ptr<TListItem<T>> partitionedEl = Partition(head);
    std::shared_ptr<TListItem<T>> leftPartition = partitionedEl->GetNext();
    std::shared_ptr<TListItem<T>> rightPartition = head;

    partitionedEl->SetNext(nullptr);

    if (leftPartition == nullptr) {
        leftPartition = head;
        rightPartition = head->GetNext();
        head->SetNext(nullptr);
    }

    rightPartition = PSort(rightPartition);
    leftPartition = PSort(leftPartition);
    std::shared_ptr<TListItem<T>> iter = leftPartition;
    while (iter->GetNext() != nullptr) {
        iter = iter->GetNext();
    }

    iter->SetNext(rightPartition);

    return leftPartition;
}

template <class T>
std::shared_ptr<TListItem<T>> TList<T>::Partition(std::shared_ptr<TListItem<T>> &head)
{
    std::lock_guard<std::mutex> lock(mutex);
    if (head->GetNext()->GetNext() = nullptr) {
        if (head->GetNext()->GetFigure()->Square() > head->GetFigure()->Square()) {
            return head->GetNext();
        } else {
            return head;
        }
    } else {
        std::shared_ptr<TListItem<T>> i = head->GetNext();
        std::shared_ptr<TListItem<T>> pivot = head;
        std::shared_ptr<TListItem<T>> lastElSwapped = (pivot->GetNext()->GetFigure()->Square() >= pivot->GetFigure()->Square()) ? pivot->GetNext() : pivot;

        while ((i != nullptr) && (i->GetNext() != nullptr)) {
            if (i->GetNext()->GetFigure()->Square() >= pivot->GetFigure()->Square()) {
                if (i->GetNext() == lastElSwapped->GetNext()) {
                    lastElSwapped = lastElSwapped->GetNext();
                } else {
                    std::shared_ptr<TListItem<T>> tmp = lastElSwapped->GetNext();
                    lastElSwapped->SetNext(i->GetNext());
                    i->SetNext(i->GetNext()->GetNext());
                    lastElSwapped = lastElSwapped->GetNext();
                    lastElSwapped->SetNext(tmp);
                }
            }
            i = i->GetNext();
        }
        return lastElSwapped;
    }
}

template <class T>
void TList<T>::Sort()
{
    if (head == nullptr)
        return;
    std::shared_ptr<TListItem<T>> tmp = head->GetNext();
    head->SetNext(PSort(tmp));
}

template <class T>
void TList<T>::ParSort()
{
    if (head == nullptr)
        return;
    std::shared_ptr<TListItem<T>> tmp = head->GetNext();
    head->SetNext(PParSort(tmp));
}

template <class T>
std::shared_ptr<TListItem<T>> TList<T>::PParSort(std::shared_ptr<TListItem<T>> &head)
{
    if (head == nullptr || head->GetNext() == nullptr) {
        return head;
    }

    std::shared_ptr<TListItem<T>> partitionedEl = Partition(head);
    std::shared_ptr<TListItem<T>> leftPartition = partitionedEl->GetNext();
    std::shared_ptr<TListItem<T>> rightPartition = head;

    partitionedEl->SetNext(nullptr);

    if (leftPartition == nullptr) {
        leftPartition = head;
        rightPartition = head->GetNext();
        head->SetNext(nullptr);
    }

    std::packaged_task<std::shared_ptr<TListItem<T>>(std::shared_ptr<TListItem<T>>&)>
        task1(std::bind(&TList<T>::PParSort, this, std::placeholders::_1));
    std::packaged_task<std::shared_ptr<TListItem<T>>(std::shared_ptr<TListItem<T>>&)>
        task2(std::bind(&TList<T>::PParSort, this, std::placeholders::_1));
    auto rightPartitionHandle = task1.get_future();
    auto leftPartitionHandle = task2.get_future();

    std::thread(std::move(task1), std::ref(rightPartition)).join();
    rightPartition = rightPartitionHandle.get();
    std::thread(std::move(task2), std::ref(leftPartition)).join();
    leftPartition = leftPartitionHandle.get();
    std::shared_ptr<TListItem<T>> iter = leftPartition;
    while (iter->GetNext() != nullptr) {
        iter = iter->GetNext();
    }

    iter->SetNext(rightPartition);
    return leftPartition;
}

\end{lstlisting}


\section{Консоль}
\begin{alltt}
karma@karma:~/mai_study/OOP/lab8$ ./run
Choose an operation:
1) Add trapeze
2) Add rhomb
3) Add pentagon
4) Delete figure from list
5) Sort list
6) Print list
0) Exit
3
Enter side: 5
Enter index = 0
Choose an operation:
1) Add trapeze
2) Add rhomb
3) Add pentagon
4) Delete figure from list
5) Sort list
6) Print list
0) Exit
2
Enter side: 5
Enter smaller angle: 30
Enter index = 0
Choose an operation:
1) Add trapeze
2) Add rhomb
3) Add pentagon
4) Delete figure from list
5) Sort list
6) Print list
0) Exit
1
Enter bigger base: 5
Enter smaller base: 4
Enter left side: 4
Enter right side: 4
Enter index = 1
Choose an operation:
1) Add trapeze
2) Add rhomb
3) Add pentagon
4) Delete figure from list
5) Sort list
6) Print list
0) Exit
3
Enter side: 10
Enter index = 2
Choose an operation:
1) Add trapeze
2) Add rhomb
3) Add pentagon
4) Delete figure from list
5) Sort list
6) Print list
0) Exit
6
idx: 0   Side = 5, smaller_angle = 30, square = 12.5, type: rhomb

idx: 1   Sides =  5, square = 43.0119, type: pentagon

idx: 2   Smaller base = 4, bigger base = 5, left side = 4, right side = 4, square = 7.96863, type: trapeze

idx: 3   Sides =  10, square = 172.048, type: pentagon


Choose an operation:
1) Add trapeze
2) Add rhomb
3) Add pentagon
4) Delete figure from list
5) Sort list
6) Print list
0) Exit
5
1 to regular sort, 2 to parallel
2
idx: 0   Smaller base = 4, bigger base = 5, left side = 4, right side = 4, square = 7.96863, type: trapeze

idx: 1   Side = 5, smaller_angle = 30, square = 12.5, type: rhomb

idx: 2   Sides =  5, square = 43.0119, type: pentagon

idx: 3   Sides =  10, square = 172.048, type: pentagon


Choose an operation:
1) Add trapeze
2) Add rhomb
3) Add pentagon
4) Delete figure from list
5) Sort list
6) Print list
0) Exit
6
idx: 0   Smaller base = 4, bigger base = 5, left side = 4, right side = 4, square = 7.96863, type: trapeze

idx: 1   Side = 5, smaller_angle = 30, square = 12.5, type: rhomb

idx: 2   Sides =  5, square = 43.0119, type: pentagon

idx: 3   Sides =  10, square = 172.048, type: pentagon

Choose an operation:
1) Add trapeze
2) Add rhomb
3) Add pentagon
4) Delete figure from list
5) Sort list
6) Print list
0) Exit
2
Enter side: 50
Enter smaller angle: 90
Enter index = 0
Choose an operation:
1) Add trapeze
2) Add rhomb
3) Add pentagon
4) Delete figure from list
5) Sort list
6) Print list
0) Exit
3
Enter side: 10000
Enter index = 0
Choose an operation:
1) Add trapeze
2) Add rhomb
3) Add pentagon
4) Delete figure from list
5) Sort list
6) Print list
0) Exit
6
idx: 0   Sides =  10000, square = 1.72048e+08, type: pentagon

idx: 1   Side = 50, smaller_angle = 90, square = 2500, type: rhomb

idx: 2   Smaller base = 4, bigger base = 5, left side = 4, right side = 4, square = 7.96863, type: trapeze

idx: 3   Side = 5, smaller_angle = 30, square = 12.5, type: rhomb

idx: 4   Sides =  5, square = 43.0119, type: pentagon

idx: 5   Sides =  10, square = 172.048, type: pentagon

Choose an operation:
1) Add trapeze
2) Add rhomb
3) Add pentagon
4) Delete figure from list
5) Sort list
6) Print list
0) Exit
5
1 to regular sort, 2 to parallel
1
idx: 0   Smaller base = 4, bigger base = 5, left side = 4, right side = 4, square = 7.96863, type: trapeze

idx: 1   Side = 5, smaller_angle = 30, square = 12.5, type: rhomb

idx: 2   Sides =  5, square = 43.0119, type: pentagon

idx: 3   Sides =  10, square = 172.048, type: pentagon

idx: 4   Side = 50, smaller_angle = 90, square = 2500, type: rhomb

idx: 5   Sides =  10000, square = 1.72048e+08, type: pentagon


Choose an operation:
1) Add trapeze
2) Add rhomb
3) Add pentagon
4) Delete figure from list
5) Sort list
6) Print list
0) Exit
0

\end{alltt}

