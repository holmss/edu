\section{Описание}

Шаблоны (template) предназначены для кодирования обобщенных алгоритмов, без привязки к некоторым параметрам (например, типам данных, размерам буферов, значениям по умолчанию). В С++ возможно создание шаблонов функций и классов. Шаблоны позволяют создавать параметризованные классы и функции. Параметром может быть любой типа или значение одного из допустимых типов (целое число, перечисляемый тип, указатель на любой объект с глобально доступным именем, ссылка). \\
Шаблоны используются в случаях дублирования одного и того же кода для нескольких типов. Например, можно использовать шаблоны функций для создания набора функций, которые применяют один и тот же алгоритм к различным типам данных. Кроме того, шаблоны классов можно использовать для разработки набора типобезопасных классов. Иногда рекомендуется использовать шаблоны вместо макросов С и пустых указателей. Шаблоны особенно полезны при работе с коллекциями и умными указателями.\\


\section{Исходный код}

\begin{longtable}{|p{7.5cm}|p{7.5cm}|}
\rowcolor{lightgray}
\multicolumn{2}{|c|} {TListItem.cpp}\\
\hline
TListItem(const std::sharedptr<T>\&obj);&Конструктор класса\\
\hline
std::sharedptr<T> GetFigure() const;&Получение фигуры из узла\\
\hline
std::sharedptr<TListItem<T>> GetNext();&Получение ссылки на следующий узел\\
\hline
std::sharedptr<TListItem<T>> GetPrev();&Получение ссылки на предыдущий узел\\
\hline
void SetNext(std::sharedptr<TListItem<T>> item);&Установка ссылки на следующий узел\\
\hline
void SetPrev(std::sharedptr<TListItem<T>> item);&Установка ссылки на предыдущий узел\\
\hline
friend std::ostream\& operator<<(std::ostream \&os, const TListItem<A> \&obj);&Переопределенный оператор вывода в поток std::ostream\\
\hline
virtual \textasciitilde TListItem(){};&Деконструктор класса\\
\hline
\rowcolor{lightgray}
\multicolumn{2}{|c|} {TList.cpp}\\
\hline
TList();&Конструктор класса\\
\hline
void Push(std::sharedptr<T> \&obj);&Добавление фигуры в список\\
\hline
std::sharedptr<T> Pop();&Получение фигуры из списка\\
\hline
const bool IsEmpty() const;&Проверка, пуст ли список\\
\hline
uint32t GetLength();&Получение длины списка\\
\hline
friend std::ostream\& operator<<(std::ostream \&os, const TList<A> \&list);&Переопределенный оператор вывода в поток std::ostream\\
\hline
virtual \textasciitilde TList();&Деконструктор класса\\
\hline
\end{longtable}


\begin{lstlisting}[language=C]

template <class T>
class TList
{
public:
    TList();
    void Push(std::shared_ptr<T> &obj);
    const bool IsEmpty() const;
    uint32_t GetLength();
    std::shared_ptr<T> Pop();
    template <class A> friend std::ostream& operator<<(std::ostream &os, const TList<A> &list);
    void Del();
    virtual ~TList();

private:
    uint32_t length;
    std::shared_ptr<TListItem<T>> head;

    void PushFirst(std::shared_ptr<T> &obj);
    void PushLast(std::shared_ptr<T> &obj);
    void PushAtIndex(std::shared_ptr<T> &obj, int32_t ind);
    std::shared_ptr<T> PopFirst();
    std::shared_ptr<T> PopLast();
    std::shared_ptr<T> PopAtIndex(int32_t ind);
};

template <class T>
class TListItem
{
public:
    TListItem(const std::shared_ptr<T> &obj);

    std::shared_ptr<T> GetFigure() const;
    std::shared_ptr<TListItem<T>> GetNext();
    std::shared_ptr<TListItem<T>> GetPrev();
    void SetNext(std::shared_ptr<TListItem<T>> item);
    void SetPrev(std::shared_ptr<TListItem<T>> item);
    template <class A> friend std::ostream& operator<<(std::ostream &os, const TListItem<A> &obj);

    virtual ~TListItem(){};

private:
    std::shared_ptr<T> item;
    std::shared_ptr<TListItem<T>> next;
    std::shared_ptr<TListItem<T>> prev;
};


\end{lstlisting}


\section{Консоль}
\begin{alltt}
karma@karma:~/mai_study/OOP/lab4$ ./run
Choose an operation:
1) Add trapeze
2) Add rhomb
3) Add pentagon
4) Delete figure from list
5) Print list
0) Exit
1
Enter bigger base: 10
Enter smaller base: 5
Enter left side: 5
Enter right side: 5
Enter index = 0
Choose an operation:
1) Add trapeze
2) Add rhomb
3) Add pentagon
4) Delete figure from list
5) Print list
0) Exit
1
Enter bigger base: 15
Enter smaller base: 10
Enter left side: 5
Enter right side: 5
Enter index = 0
Choose an operation:
1) Add trapeze
2) Add rhomb
3) Add pentagon
4) Delete figure from list
5) Print list
0) Exit
2
Enter side: 10
Enter smaller angle: 60
Enter index = 1
Choose an operation:
1) Add trapeze
2) Add rhomb
3) Add pentagon
4) Delete figure from list
5) Print list
0) Exit
3
Enter side: 3
Enter index = 2
Choose an operation:
1) Add trapeze
2) Add rhomb
3) Add pentagon
4) Delete figure from list
5) Print list
0) Exit
5
idx: 0   Smaller base = 10, bigger base = 15, left side = 5, right side = 5, type: trapeze

idx: 1   Smaller base = 5, bigger base = 10, left side = 5, right side = 5, type: trapeze

idx: 2   Side = 10, smaller_angle = 60, type: rhomb

idx: 3   Sides =  3, type: pentagon


Choose an operation:
1) Add trapeze
2) Add rhomb
3) Add pentagon
4) Delete figure from list
5) Print list
0) Exit
4
Enter index = 0
Choose an operation:
1) Add trapeze
2) Add rhomb
3) Add pentagon
4) Delete figure from list
5) Print list
0) Exit
4
Enter index = 1
Choose an operation:
1) Add trapeze
2) Add rhomb
3) Add pentagon
4) Delete figure from list
5) Print list
0) Exit
4
Enter index = 1
Choose an operation:
1) Add trapeze
2) Add rhomb
3) Add pentagon
4) Delete figure from list
5) Print list
0) Exit
4
Enter index = 0
Choose an operation:
1) Add trapeze
2) Add rhomb
3) Add pentagon
4) Delete figure from list
5) Print list
0) Exit
5
The list is empty.

Choose an operation:
1) Add trapeze
2) Add rhomb
3) Add pentagon
4) Delete figure from list
5) Print list
0) Exit
0

\end{alltt}

