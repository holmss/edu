
%% * Copyright (c) w495, 2009 
%% *
%% * All rights reserved.
%% *
%% * Redistribution and use in source and binary forms, with or without
%% * modification, are permitted provided that the following conditions are met:
%% *     * Redistributions of source code must retain the above copyright
%% *       notice, this list of conditions and the following disclaimer.
%% *     * Redistributions in binary form must reproduce the above copyright
%% *       notice, this list of conditions and the following disclaimer in the
%% *       documentation and/or other materials provided with the distribution.
%% *     * Neither the name of the w495 nor the
%% *       names of its contributors may be used to endorse or promote products
%% *       derived from this software without specific prior written permission.
%% *
%% * THIS SOFTWARE IS PROVIDED BY w495 ''AS IS'' AND ANY
%% * EXPRESS OR IMPLIED WARRANTIES, INCLUDING, BUT NOT LIMITED TO, THE IMPLIED
%% * WARRANTIES OF MERCHANTABILITY AND FITNESS FOR A PARTICULAR PURPOSE ARE
%% * DISCLAIMED. IN NO EVENT SHALL w495 BE LIABLE FOR ANY
%% * DIRECT, INDIRECT, INCIDENTAL, SPECIAL, EXEMPLARY, OR CONSEQUENTIAL DAMAGES
%% * (INCLUDING, BUT NOT LIMITED TO, PROCUREMENT OF SUBSTITUTE GOODS OR SERVICES;
%% * LOSS OF USE, DATA, OR PROFITS; OR BUSINESS INTERRUPTION) HOWEVER CAUSED AND
%% * ON ANY THEORY OF LIABILITY, WHETHER IN CONTRACT, STRICT LIABILITY, OR TORT
%% * (INCLUDING NEGLIGENCE OR OTHERWISE) ARISING IN ANY WAY OUT OF THE USE OF THIS
%% * SOFTWARE, EVEN IF ADVISED OF THE POSSIBILITY OF SUCH DAMAGE.

\documentclass[pdf, unicode, 12pt, a4paper,oneside,fleqn]{article}
\usepackage[utf8]{inputenc}
\usepackage[T2B]{fontenc}
\usepackage[english,russian]{babel}

\frenchspacing

\usepackage{amsmath}
\usepackage{amssymb}
\usepackage{hyperref}
\usepackage{longtable}
\usepackage[table]{xcolor}  
\usepackage{array}
\usepackage{color}
\usepackage{xcolor}

\usepackage{hyperref}


\newcommand{\MYhref}[3][blue]{\href{#2}{\color{#1}{#3}}}%

\usepackage{listings}
\usepackage{alltt}
\usepackage{csquotes}
\DeclareQuoteStyle{russian}
	{\guillemotleft}{\guillemotright}[0.025em]
	{\quotedblbase}{\textquotedblleft}
\ExecuteQuoteOptions{style=russian}

\usepackage{graphicx}

\usepackage{listings}
\lstset{tabsize=2,
	breaklines,
	columns=fullflexible,
	flexiblecolumns,
	numbers=left,
	numberstyle={\footnotesize},
	extendedchars,
	inputencoding=utf8}

\usepackage{longtable}

\def\@xobeysp{ }
\def\verbatim@processline{\hspace{1.2cm}\raggedright\the\verbatim@line\par}

\oddsidemargin=-0.4mm
\textwidth=160mm
\topmargin=4.6mm
\textheight=210mm

\parindent=0pt
\parskip=3pt

\definecolor{lightgray}{gray}{0.9}


\renewcommand{\thesubsection}{\arabic{subsection}}

\lstdefinestyle{customc}{
  belowcaptionskip=1\baselineskip,
  breaklines=true,
  frame=L,
  xleftmargin=\parindent,
  language=C,
  showstringspaces=false,
  basicstyle=\footnotesize\ttfamily,
  keywordstyle=\bfseries\color{green!40!black},
  commentstyle=\itshape\color{gray},
  identifierstyle=\color{black},
  stringstyle=\color{blue},
}

\lstdefinestyle{customasm}{
  belowcaptionskip=1\baselineskip,
  frame=L,
  xleftmargin=\parindent,
  language=[x86masm]Assembler,
  basicstyle=\footnotesize\ttfamily,
  commentstyle=\itshape\color{purple!40!black},
}

\lstset{escapechar=@,style=customc}


\newcommand{\CWPHeader}[1]{\addtocounter{section}{-1}\section{#1}}

% Заголовок курсовой работы
% Единственный аргумент --- ее тема
\newcommand{\CWHeader}[1]{\section*{#1}}

\newcommand{\CWProblem}[1]{\par\textbf{Задача: }#1}

\begin{document}

\begin{center}
\bfseries

{\Large Московский авиационный институт\\ (национальный исследовательский университет)

}

\vspace{48pt}

{\large Факультет информационных технологий и прикладной математики
}

\vspace{36pt}


{\large Кафедра вычислительной математики и~программирования

}


\vspace{48pt}

Лабораторная работа №1 по курсу \enquote{Дискретный анализ}

\end{center}

\vspace{72pt}

\begin{flushright}
\begin{tabular}{rl}
Студент: & М.\,В Спиридонов \\
Преподаватель: & А.\,А. Кухтичев \\
Группа: & М8О-206Б \\
Дата: & \\
Оценка: & \\
Подпись: & \\
\end{tabular}
\end{flushright}

\vfill
\pagestyle{empty}
\begin{center}
\bfseries
Москва, \the\year
\end{center}
\pagebreak

\CWHeader{Лабораторная работа \textnumero 1}

\CWProblem{ 
Требуется разработать программу, осуществляющую ввод пар \enquote{ключ-значение}, их 
упорядочивание по возрастанию ключа указанным алгоритмом сортировки за линейное время и вывод отсортированной последовательности.

{\bfseries Вариант сортировки:} { Сортировка подсчётом.}

{\bfseries Вариант ключа:} {  Числа от 0 до 65535. }

{\bfseries Вариант значения:} {  Сортировка подсчётом по целочисленному ключу (числа от 0 до 65535).
В качестве значения используются строки переменной длины (до 2048 символов)}
}
\pagestyle{plain}
\pagebreak

\section{Описание}
Условие задачи подразумевает хранение строк, но использовать STL было запрещено и пришлось написать свой класс TString, который реализует все необходимые функции от строки (сравнение, копирование и присваивание). По сути была написана лишь красивая обёртка над char*.\newline
Также нужно было как-то хранить пары \textit{\enquote{ключ-значение}}, поэтому был реализован свой класс TPair, который содержит два публичных поля: Key(int), Value(TString).\newline
Из основных элементов не хватало только динамического массива, поэтому был реализован свой TVector, который позволяет динамически добавлять элементы TPair и производить манипуляции с ними. \newline
Все основные типы данных есть, осталась сортировка подсчётом и ввод данных.\newline
Есть смысл сказать о методе ввода значений. Был создан буфер char* в 2048 символов, куда при помощи scanf("\%s") считывалась строка до EOF, и позднее заносилась в TPair, который в свою очередь попадал в TVector.\newline
Сортировка подсчётом имеет алгоритмическую сложность $O(n+k)$. \newline Применима в случае, если сортируемые числа имеют (или их можно отобразить в) диапазон возможных значений, который достаточно мал по сравнению с сортируемым множеством. \newline 
Суть сортировки заключается в нахождении максимального ключа среди всех входных и создании дополнительного целочисленного массива, где будет подсчитано сколько каких ключей встречено. Далее в зависимости от значения ключа, в результирующий массив будут расставлены все исходные пары.\newline
Я использовал стабильную версию сортировки для структур, рассмотренной на лекции. 
\pagebreak

\section{Описание программы}
По условию задачи было ясно, что красиво такую программу написать в одном файле не получится и придется её разбить на несколько основных файлов:
\begin{enumerate}
\item main.cpp (содержит основной метод main, метод вывода TVector'а - Print)
\item Model.h и Model.cpp (описание и реализация класса TPair)
\item TString.h и TString.cpp (описание и реализация класса TString)
\item TVector.h и TVector.cpp (описание и реализация класса TVector)
\end{enumerate}
Для удобства сборки проекта были написаны: 
\begin{enumerate}
\item makefile (для сборки и автоматического тестирования проекта из консоли)
\item CMakeLists.txt (для автоматической сборки проекта в CLion, Visual Studio Code, Visual Studio 17)
\end{enumerate}
Также были взяты "test\_generator.py" и "wrapper.sh" c GitHub'а[1] и переделаны.
\newpage
\begin{longtable}{|p{8.4cm}|p{7.5cm}|}
\hline
\rowcolor{lightgray}
\multicolumn{2}{|c|} {main.cpp}\\
\hline
void Print (TVector \&inp)&Метод для вывода вектора\\
\hline
\hline
int main (int argc, char **argv)&Точка входа в программу\\
\hline
\rowcolor{lightgray}
\multicolumn{2}{|c|} {Model.h (описание методов/полей)}\\
\hline
TPair ()&Пустой конструктор\\
TPair (const TPair \&that)&Конструктор копирования\\
TPair (int key, TString value)&Конструктор создания с параметрами\\
\hline
\hline
TPair \&operator = (const TPair \&that)&Оператор присваивания\\
\hline
\hline \textasciitilde TPair ()&Деструктор\\ \hline
\hline int Key&Поле ключа\\ \hline
\hline
TString Value&Поле значения\\
\hline
\rowcolor{lightgray}
\multicolumn{2}{|c|} {TString.h (описание методов/полей)}\\
\hline
TString (const char *string = "")&Конструктор создания из char*\\
TString (TString const \&string)&Конструктор корпирования\\
\hline
\hline
TString \&operator = (const TString \&string)&Оператор присваивания\\
TString \&operator = (const char *string)&Оператор присваивания\\
\hline
\hline bool operator == (const TString \&string) const&Оператор сравнения\\ \hline
\hline char \&operator [] (int) const&Оператор обращения по индексу\\ \hline
\hline char \&operator [] (int)&Оператор обращения по индексу\\ \hline
\hline const size\_t Length () const&Метод, который возвращает длину строки \\ \hline
\hline char *GetCharArray ()&Метод получения внутреннего char*\\ \hline
\hline  char *GetCharArray () const&Метод получения внутреннего char*\\ \hline
\hline \textasciitilde TString()&Деструктор \\ \hline
\hline static int const SIZE\_DIFFERENCE = 1&Приватная константа \\ \hline
\hline char *privateString&Приватное поле типа char*\\ \hline
\hline static int const DEFAULT\_INT\_VALUE = 0&Приватная константа\\ \hline
\end{longtable}
\newpage
\begin{longtable}{|p{8.4cm}|p{7.52cm}|}
\hline
\rowcolor{lightgray}
\multicolumn{2}{|c|} {TVector.h (описание методов/полей)}\\
\hline 
TVector ()&Пустой конструктор\\
TVector (const TVector\&inp)&Конструктор копирования\\
TVector (const int n)&Конструктор создания\\
\hline
\hline
void FastPushBack (TPair \&data)&Метод добавления элемента\\
\hline
\hline
void Clear ()&Метод опустошения вектора\\
\hline
\hline
int Size () const&Метод, который возвращает размер вектора\\
\hline
\hline
int Capacity () const&Метод, который возвращает реальное количество элементов в векторе\\
\hline
\hline
void CountingSort()&Метод сортировки подсчётом\\
\hline
\hline
\textasciitilde TVector ()&Деструктор\\
\hline
\hline
TPair *privateArray&Приватное поле вектора\\
\hline
\hline
int privateArraySize&Приватное поле размера вектора\\
\hline
\hline
int privateArrayOccupiedSize&Приватное поле, которое содержит количество реально занятых ячеек вектора\\
\hline
\hline
static int const SIZE\_DIFFERENCE = 1&Приватная константа\\
\hline
\hline
static int const DEFAULT\_INT\_VALUE = 0&Поле значения\\
\hline
\end{longtable}
\newpage
\section{Код программы}
Метод сортировки подсчётом
\begin{lstlisting}
void TVector::CountingSort () {

    //clock_t start = clock();
    int max = 0;
    for (int i = 0; i < privateArrayOccupiedSize; i++) {
        if (privateArray[i].Key > max) {
            max = (privateArray)[i].Key;
        }
    }
    TPair *res = new TPair[privateArrayOccupiedSize];

    int *countMass = new int[max + 1]{0};
    for (int i = 0; i < privateArrayOccupiedSize; ++i) {
        countMass[privateArray[i].Key]++;
    }

    for (int i = 1; i <= max; ++i) {
        countMass[i] = countMass[i] + countMass[i - 1];
    }

    for (int i = privateArrayOccupiedSize - 1; i >= 0; i--) {
        int tmp = countMass[privateArray[i].Key] - 1;
        res[tmp] = privateArray[i];
        countMass[privateArray[i].Key] = countMass[privateArray[i].Key] - 1;
    }
    for (int i = 0; i < privateArrayOccupiedSize; ++i) {
        privateArray[i] = res[i];
    }
    delete[] countMass;
    delete[] res;

    //clock_t end = clock();
    //printf("\nCounter SortTime = %lf\n", (double)(end - start) / CLOCKS_PER_SEC);
}
\end{lstlisting}

\newpage
\section{Код дополнительных файлов}
\lstset{language=BASH}
\begin{itemize}
\item makefile
\begin{lstlisting}
FLAGS=-g -std=c++11 -pedantic -Wall -Werror -Wno-sign-compare -Wno-long-long -O2
CC=g++

all: string model vector main

main: main.cpp
	$(CC) $(FLAGS) -c main.cpp
	$(CC) $(FLAGS) -o CounterSort TString.o Model.o TVector.o main.o -lm

vector: TVector.cpp
	$(CC) $(FLAGS) -c TVector.cpp

string: TString.cpp
	$(CC) $(FLAGS) -c TString.cpp

model: Model.cpp
	$(CC) $(FLAGS) -c Model.cpp
	
clear:
	rm -f *.o
	rm -fr *.dSYM

runtests:
	rm -rf tests
	mkdir tests
	sh wrapper.sh

bench: benchmark.cpp
	$(CC) $(FLAGS) -c benchmark.cpp
	$(CC) $(FLAGS) -o benchmark benchmark.o -lm	
\end{lstlisting}
\item test\_generator.py
\lstset{language=Python}
\begin{lstlisting}
import random
import string

MAX_KEY_VALUE = 65535
MAX_VALUE_LEN = 500

def generate_random_value():
    return "".join([random.choice(string.ascii_lowercase) 
    	for _ in range(1, MAX_VALUE_LEN)])

if __name__ == "__main__":
    for num in range(1, 21):
        values = list()
        output_filename = "tests/{:02d}.t".format(num)
        with open(output_filename, 'w') as output:
            for _ in range(0, random.randint(10, 100)):
                key = random.randint(0, MAX_KEY_VALUE)
                value = generate_random_value()
                values.append((key, value))
                output.write("{}\t{}\n".format(key, value))
        # Answer.
        # values[0][0] -- key
        # values[0][1] -- value
        output_filename = "tests/{:02d}.a".format(num)
        with open(output_filename, 'w') as output:
            values = sorted(values, key=lambda x: x[0])
            for value in values:
                output.write("{}\t{}\n".format(value[0], value[1]))
\end{lstlisting}
\item benchmark.cpp
\lstset{language=C}
\begin{lstlisting}
#include <algorithm>
#include <cassert>
#include <ctime>
#include <iostream>
#include <vector>

bool CheckOrder (const std::vector <std::pair <int, std::string> > &data) {
    for (size_t idx = 1; idx < data.size(); ++idx) {
        if (data[idx - 1].first > data[idx].first) {
            return false;
        }
    }
    return true;
}

bool PairComparer (const std::pair <int, std::string> &a, std::pair <int, std::string> &b) {
    return a.first < b.first;
}
void CountingSort (std::vector <std::pair <int, std::string> >&inp) {
    int size = inp.capacity();
    int max = 0;
    for (int i = 0; i < size; i++) {
        if (inp[i].first > max) {
            max = inp[i].first;
        }
    }
    std::vector <std::pair <int, std::string> >res(size);
    int *countMass = new int[max + 1];
    for (int i = 0; i < max + 1; ++i) {
        countMass[i] = 0;
    }
    for (int i = 0; i < size; ++i) {
        countMass[inp[i].first]++;
    }

    for (int i = 1; i <= max; ++i) {
        countMass[i] = countMass[i] + countMass[i - 1];
    }

    for (int i = size - 1; i >= 0; i--) {
        int tmp = countMass[(inp)[i].first] - 1;
        res[tmp] = inp[i];
        countMass[inp[i].first] = countMass[inp[i].first] - 1;
    }
    for (int i = 0; i < size; ++i) {
        inp[i] = res[i];
    }
    delete[] countMass;
}
void StdSort (std::vector <std::pair <int, std::string> > &data) {
    std::sort(data.begin(), data.end(), PairComparer);
}

void RunStdSort (std::vector <std::pair <int, std::string> > &data) {
    clock_t start = clock();
    StdSort(data);
    clock_t end = clock();
    assert(CheckOrder(data) == true);
    double time = (double)(end - start);
    printf("\nStd SortTime = %lf\n",time/CLOCKS_PER_SEC);
}
void RunCountingSort (std::vector <std::pair <int, std::string> > &data) {
    clock_t start = clock();
    CountingSort(data);
    clock_t end = clock();
    assert(CheckOrder(data) == true);
    double time = (double)(end - start);
    printf("\nCounting SortTime = %lf\n",time/CLOCKS_PER_SEC);
}

int main (int argc, const char *argv[]) {
    std::vector <std::pair <int, std::string> > data;
    int tempInt = 0;
    char *bufStr = new char[2048];
    std::pair <int, std::string> pair;
    while (scanf("%d", &tempInt) != EOF) {
        scanf("%s", bufStr);
        pair.first = tempInt;
        pair.second = bufStr;

        data.push_back(pair);
    }
    delete[] bufStr;
    std::vector <std::pair <int, std::string> > dataCopy(data);
    RunStdSort(data);
    RunCountingSort(data);
    return 0;
}

\end{lstlisting}

\end{itemize}
\newpage
\section{Консоль}
\lstset{language=Bash}
\begin{alltt}
$ CounterSort make
g++ -g -std=c++11 -pedantic -Wall -Werror -Wno-sign-compare 
-Wno-long-long -O2 -c TString.cpp
g++ -g -std=c++11 -pedantic -Wall -Werror -Wno-sign-compare 
-Wno-long-long -O2 -c Model.cpp
g++ -g -std=c++11 -pedantic -Wall -Werror -Wno-sign-compare 
-Wno-long-long -O2 -c TVector.cpp
g++ -g -std=c++11 -pedantic -Wall -Werror -Wno-sign-compare 
-Wno-long-long -O2 -c main.cpp
g++ -g -std=c++11 -pedantic -Wall -Werror -Wno-sign-compare 
-Wno-long-long -O2 -o CounterSort TString.o Model.o TVector.o main.o -lm
\end{alltt}
\lstset{language=BASH}
\begin{alltt}
$ CounterSort cat 01.t
60693	nmbfyzfqu
43600	piqyjgkhf
25976	irudfyioj
3679	duqjersbu
38028	wuxtiiogk
3639	mefjaasbe
3915	loxlyyqcs
42023	sslttackw
58339	zoxkqedkb
$ CounterSort ./CounterSort < 01.t > 01.answ
$ CounterSort cat 01.answ
3639	mefjaasbe
3679	duqjersbu
3915	loxlyyqcs
25976	irudfyioj
38028	wuxtiiogk
42023	sslttackw
43600	piqyjgkhf
58339	zoxkqedkb
60693	nmbfyzfqu
$ CounterSort cat 02.t
14517	tyxzddarc
64044	xelwhtsro
39337	kwktcfiiu
9162	qjevarqqs
64981	mrwlghlcs
9480	vagjwzilk
44021	egaeqezha
41017	xpcfbjuly
51997	jyjntrrgf
41622	pkabowgux
44235	yborezsfg
$ CounterSort ./CounterSort < 02.t > 02.answ
$ CounterSort cat 02.answ
9162	qjevarqqs
9480	vagjwzilk
14517	tyxzddarc
39337	kwktcfiiu
41017	xpcfbjuly
41622	pkabowgux
44021	egaeqezha
44235	yborezsfg
51997	jyjntrrgf
64044	xelwhtsro
64981	mrwlghlcs
\end{alltt}
\newpage
\section{Проверка на утечки памяти }
Тестирование проводилось на файле в 100 000 строк, на системе Arch Linux, с помощью утилиты Valgrind.
\begin{alltt}
[Ср окт 04: 05:53] snowden@arch ~/Projects/DA/da1 [master] $ make
g++ -g -std=c++11 -pedantic -Wall -Werror -Wno-sign-compare 
-Wno-long-long -O2 -c TString.cpp
g++ -g -std=c++11 -pedantic -Wall -Werror -Wno-sign-compare 
-Wno-long-long -O2 -c Model.cpp
g++ -g -std=c++11 -pedantic -Wall -Werror -Wno-sign-compare 
-Wno-long-long -O2 -c TVector.cpp
g++ -g -std=c++11 -pedantic -Wall -Werror -Wno-sign-compare 
-Wno-long-long -O2 -c main.cpp
g++ -g -std=c++11 -pedantic -Wall -Werror -Wno-sign-compare 
-Wno-long-long -O2 -o CounterSort TString.o Model.o TVector.o main.o -lm
[Ср окт 04: 05:54] snowden@arch ~/Projects/DA/da1 [master] $ 
valgrind ./CounterSort < tests/01.t > tmp 
==1417== Memcheck, a memory error detector 
==1417== Copyright (C) 2002-2017, and GNU GPL'd, by Julian Seward et al.
==1417== Using Valgrind-3.13.0 and LibVEX; rerun with -h for copyright info
==1417== Command: ./CounterSort
==1417== 
==1417== 
==1417== HEAP SUMMARY:
==1417== in use at exit: 0 bytes in 0 blocks
==1417== total heap usage: 1,517,400 allocs, 1,517,400 frees, 9,380,563 bytes 
allocated
==1417== 
==1417== All heap blocks were freed -- no leaks are possible
==1417== 
==1417== For counts of detected and suppressed errors, rerun with: -v
==1417== ERROR SUMMARY: 0 errors from 0 contexts (suppressed: 0 from 0)
[Ср окт 04: 05:54] snowden@arch ~/Projects/DA/da1 [master] $ wc tests/01.t 
100000 200000 1083130 tests/01.t
\end{alltt}
\pagebreak

\section{Тест производительности}
Чтобы увидеть разницу был созданы файлы размером $10^3$, $10^5$, $10^{6}$, $10*10^{6}$ входных строк и был создан файл бенчмарка, с реализацией сортировки подсчетом и QSort из STD. Исходный код лежит в \enquote{дополнительных файлах} выше.

\begin{alltt}
$ CounterSort make bench
g++ -g -std=c++11 -pedantic -Wall -Werror -Wno-sign-compare -Wno-long-long -O2 
-c benchmark.cpp
g++ -g -std=c++11 -pedantic -Wall -Werror -Wno-sign-compare -Wno-long-long -O2 
-o benchmark benchmark.o -lm
$ CounterSort wc tests/01.t
	1000    2000   15797 tests/01.t
$ CounterSort ./benchmark < tests/01.t
Std SortTime = 0.000080
Counting SortTime = 0.000241

$ CounterSort wc tests/02.t
	100000  200000 1083096 tests/02.t
$ CounterSort ./benchmark < tests/02.t 
Std SortTime = 0.010439
Counting SortTime = 0.005754

$ CounterSort wc tests/03.t
	100000  200000 1083096 tests/03.t
$ CounterSort ./benchmark < tests/03.t 
Std SortTime = 0.095891
Counting SortTime = 0.050415

$ CounterSort wc tests/04.t
	10000000 20000000 108302115 tests/04.t
Std SortTime = 0.984407
Counting SortTime = 0.854098
\end{alltt}
\pagebreak
\section{Дневник отладки}
\begin{enumerate}
\item 29.09.17; 0:07; Сел за написание лабораторной. Написал TString, TPair, TVector. Долго искал информацию по грамотному разделению шаблона на несколько файлов;
\item 1.10.17; 0:49; Первый провал на системе тестирования, убрал rm из makefile;
\item 1.10.17; 1:04; Неверный ответ на первом тесте, программа теряла где-то одно значение. Забыл убрать отладочную строчку с удалением корневого элемента;
\item 1.10.17; 23:24; Превышение временного порога на 10-м тесте. На самом деле я даже не удивился. Изменять размер вектора при каждом добавлении элемента само по себе не очень решение. Исправлял введением доп. переменной в вектор, которая считала сколько всего элементов реально заполнены и в случае её равенства размеру массива, увеличивал размер в 2 раза;
\item 2.10.17; 2:37; Решил дописать кучу никому не нужных методов и оберток, чтобы не нарушать инкапсуляцию;
\item 2.10.17; 4:55; Превышение временного порога на 12-м тесте. Поправил начальный размер вектора с 1-го элемента до 32-х;
\item 2.10.17; 4:59; Программа получила статус \enquote{Ок} на системе автоматического тестирования;
\item 2.10.17; 18:52; Новая версия программы с исправлениями утечек памяти провалила по времени 12-й тест, но это лишь потому что я решил попробовать заслать с начальным массивом в 8 элементов. Также запускал на разных машинках, дописывал код;
\item 2.10.17; 21:41; Решил самоутвердиться и заслал финальную версию программы, которую тестировал пару-тройку часов;
\item 3.10.17; 5:57; Финальная отправка на проверку, перед отправкой отчёта. Внесены мелкие правки по кодстайлу;
\item 3.10.17; 6:27; Засылаю отчёт;
\item 6.10.17; 14:54; Исправил отчёт. Нашёл ошибку в бенчмарке - он показывал не правильное время. Исправил замеры времени, также исправил нарушение инкапсуляции. Пришлось отказаться от шаблона в виде TVector.h+TVector.hpp и написать лишь одну его реализацию, которая зависит только от TPair. Также метод сортировки был перенесён в TVector.cpp.
 \end{enumerate}
 \pagebreak
\section{Выводы}

Сортировка подсчётом применима, если сортируемые числа имеют диапазон возможных значений, который достаточно мал по сравнению с сортируемым множеством. Само написание программы не вызвало особых затруднений. Куда больше сил и времени было потрачено на всевозможные тесты и приведение кода в презентабельный вид.\newline
Весьма много времени ушло на сравнение моей сортировки с другими. Для себя я выяснил, что тестировать все на железе i7 4/8 + SSD (1400мб/1200мб чтение/запись в сек) глупая затея, ибо чтобы увидеть хоть какую-то разницу нужно создавать входные файлы $>10^{18}$ строк. Также написание лабораторной мне показалось чем-то схожим с хакатонами, которые я регулярно посещаю: быстро и красиво написать что-то почти не реально. Написанный за ночь код, нужно доводить еще пару тройку дней до совершенства, проводя уйму времени за тестами.
\newline
На вооружение забрал себе Radix, включающий в себя сортировку подсчётом. Потому что все проведённые мною тесты показали лучшие результаты именно на нем.
\pagebreak
\begin{thebibliography}{99}
\bibitem{github_t}
{\itshape GitHub одного из семинаристов.} \\URL: \texttt{https://github.com/toshunster/da} (дата обращения: 29.09.2017). 
\bibitem{wikipedia_sort}
{\itshape Шаблоны функций в С++.} \\URL: \texttt{http://cppstudio.com/post/5165/} (дата обращения: 30.09.2017). 
\bibitem{wikipedia_sort}
{\itshape Шаблоны функций в С++.} \\URL: \texttt{http://cppstudio.com/post/673/} (дата обращения: 1.10.2017). 
\bibitem{wikipedia_sort}
{\itshape Функция strcpy.} \\URL: \texttt{http://cppstudio.com/post/686/} (дата обращения: 1.10.2017). 
\bibitem{wikipedia_sort}
{\itshape Сортировка подсчётом — Википедия.} \\URL: \texttt{http://ru.wikipedia.org/wiki/Сортировка\_подсчётом} (дата обращения: 16.12.2013). 
\bibitem{Kormen}
Томас\,Х.\,Кормен, Чарльз\,И.\,Лейзерсон, Рональд\,Л.\,Ривест, Клиффорд\,Штайн.
{\itshape Алгоритмы: построение и анализ, 2-е издание.} --- Издательский дом \enquote{Вильямс}, 2007. Перевод с английского: И.\,В.\,Красиков, Н.\,А.\,Орехова, В.\,Н.\,Романов. --- 1296 с. (ISBN 5-8459-0857-4 (рус.))
\bibitem{wikipedia_sort}
{\itshape Bitbucket лектора.} \\URL: \texttt{https://bitbucket.org/nkmakarov/da4students/} (дата обращения: 2.10.2017). \end{thebibliography}
\end{document}
