
%% * Copyright (c) w495, 2009 
%% *
%% * All rights reserved.
%% *
%% * Redistribution and use in source and binary forms, with or without
%% * modification, are permitted provided that the following conditions are met:
%% *     * Redistributions of source code must retain the above copyright
%% *       notice, this list of conditions and the following disclaimer.
%% *     * Redistributions in binary form must reproduce the above copyright
%% *       notice, this list of conditions and the following disclaimer in the
%% *       documentation and/or other materials provided with the distribution.
%% *     * Neither the name of the w495 nor the
%% *       names of its contributors may be used to endorse or promote products
%% *       derived from this software without specific prior written permission.
%% *
%% * THIS SOFTWARE IS PROVIDED BY w495 ''AS IS'' AND ANY
%% * EXPRESS OR IMPLIED WARRANTIES, INCLUDING, BUT NOT LIMITED TO, THE IMPLIED
%% * WARRANTIES OF MERCHANTABILITY AND FITNESS FOR A PARTICULAR PURPOSE ARE
%% * DISCLAIMED. IN NO EVENT SHALL w495 BE LIABLE FOR ANY
%% * DIRECT, INDIRECT, INCIDENTAL, SPECIAL, EXEMPLARY, OR CONSEQUENTIAL DAMAGES
%% * (INCLUDING, BUT NOT LIMITED TO, PROCUREMENT OF SUBSTITUTE GOODS OR SERVICES;
%% * LOSS OF USE, DATA, OR PROFITS; OR BUSINESS INTERRUPTION) HOWEVER CAUSED AND
%% * ON ANY THEORY OF LIABILITY, WHETHER IN CONTRACT, STRICT LIABILITY, OR TORT
%% * (INCLUDING NEGLIGENCE OR OTHERWISE) ARISING IN ANY WAY OUT OF THE USE OF THIS
%% * SOFTWARE, EVEN IF ADVISED OF THE POSSIBILITY OF SUCH DAMAGE.

\documentclass[pdf, unicode, 12pt, a4paper,oneside,fleqn]{article}
\usepackage[utf8]{inputenc}
\usepackage[T2B]{fontenc}
\usepackage[english,russian]{babel}

\frenchspacing

\usepackage{amsmath}
\usepackage{amssymb}
\usepackage{hyperref}
\usepackage{longtable}
\usepackage[table]{xcolor}  
\usepackage{array}
\usepackage{color}
\usepackage{xcolor}

\usepackage{hyperref}


\newcommand{\MYhref}[3][blue]{\href{#2}{\color{#1}{#3}}}%

\usepackage{listings}
\usepackage{alltt}
\usepackage{csquotes}
\DeclareQuoteStyle{russian}
	{\guillemotleft}{\guillemotright}[0.025em]
	{\quotedblbase}{\textquotedblleft}
\ExecuteQuoteOptions{style=russian}

\usepackage{graphicx}

\usepackage{listings}
\lstset{tabsize=2,
	breaklines,
	columns=fullflexible,
	flexiblecolumns,
	numbers=left,
	numberstyle={\footnotesize},
	extendedchars,
	inputencoding=utf8}

\usepackage{longtable}

\def\@xobeysp{ }
\def\verbatim@processline{\hspace{1.2cm}\raggedright\the\verbatim@line\par}

\oddsidemargin=-0.4mm
\textwidth=160mm
\topmargin=4.6mm
\textheight=210mm

\parindent=0pt
\parskip=3pt

\definecolor{lightgray}{gray}{0.9}


\renewcommand{\thesubsection}{\arabic{subsection}}

\lstdefinestyle{customc}{
  belowcaptionskip=1\baselineskip,
  breaklines=true,
  frame=L,
  xleftmargin=\parindent,
  language=C,
  showstringspaces=false,
  basicstyle=\footnotesize\ttfamily,
  keywordstyle=\bfseries\color{green!40!black},
  commentstyle=\itshape\color{gray},
  identifierstyle=\color{black},
  stringstyle=\color{blue},
}

\lstdefinestyle{customasm}{
  belowcaptionskip=1\baselineskip,
  frame=L,
  xleftmargin=\parindent,
  language=[x86masm]Assembler,
  basicstyle=\footnotesize\ttfamily,
  commentstyle=\itshape\color{purple!40!black},
}

\lstset{escapechar=@,style=customc}


\newcommand{\CWPHeader}[1]{\addtocounter{section}{-1}\section{#1}}

% Заголовок курсовой работы
% Единственный аргумент --- ее тема
\newcommand{\CWHeader}[1]{\section*{#1}}

\newcommand{\CWProblem}[1]{\par\textbf{Задача: }#1}

\begin{document}

\begin{center}
\bfseries

{\Large Московский авиационный институт\\ (национальный исследовательский университет)

}

\vspace{48pt}

{\large Факультет информационных технологий и прикладной математики
}

\vspace{36pt}


{\large Кафедра вычислительной математики и~программирования

}


\vspace{48pt}

Лабораторная работа №2 по курсу \enquote{Дискретный анализ}

\end{center}

\vspace{72pt}

\begin{flushright}
\begin{tabular}{rl}
Студент: & М.\,В Спиридонов \\
Преподаватель: & А.\,А. Кухтичев \\
Группа: & М8О-206Б \\
Дата: & \\	
Оценка: & \\
Подпись: & \\
\end{tabular}
\end{flushright}

\vfill
\setcounter{page}{0}
\pagestyle{empty}
\begin{center}
\bfseries
Москва, \the\year
\end{center}
\pagebreak

\CWHeader{Лабораторная работа \textnumero 2}

\CWProblem{ 
Необходимо создать программную библиотеку, реализующую указанную структуру данных, на основе которой разработать программу-словарь. В словаре каждому ключу, представляющему из себя регистронезависимую последовательность букв английского алфавита длиной не более 256 символов, поставлен в соответствие некоторый номер, от 0 до $2^{64} - 1$. Разным словам может быть поставлен в соответствие один и тот же номер.

{\bfseries Вариант дерева:} { PATRICIA.}

{\bfseries Вариант ключа:} { регистронезависимая последовательность букв английского алфавита длиной не более 256 символов. }

{\bfseries Вариант значения:} { числа от 0 до $2^{64} - 1$.}
}
\pagestyle{plain}
\pagebreak

\section{Описание}
Нагруженное дерево (или Trie; Patricia - разновидность Trie-дерева) -- структура данных реализующая интерфейс ассоциативного массива, то есть позволяющая хранить пары \enquote{ключ--значение}. В большинстве случаев ключами выступают строки, однако в качестве ключей можно использовать любые типы данных, представимые как последовательность байт.\\ \\
В узлах Trie хранятся односимвольные метки, а ключём, который соответствует некоему узлу является путь от корня дерева до этого узла.\\
Корень дерева, очевидно, соответствует пустому ключу.\\ 

Cжатое префиксное дерево Patricia является довольно быстрым и эффективным по памяти методом реализации словаря. Применение этой модели дерева может существенно увеличить продуктивность поиска и внесения элементов в словарь. Также есть не мало алгоритмов, которые построены на принципе дерева Patricia, например алгоритм поиска подстроки \enquote{Ахо — Корасик}.

Существует 2 основных типа оптимизации нагруженного дерева:
\begin{enumerate}
\item Сжатое нагруженное дерево получается из обычного удалением промежуточных узлов, которые ведут к единственному не промежуточному узлу. Например, цепочка промежуточных узлов с метками a, b, c заменяется на один узел с меткой abc.
\item Patricia нагруженное дерево получается из сжатого (или обычного) удалением промежуточных узлов, которые имеют одного ребенка.
\end{enumerate}
Рассмотрим основные операции, связанные с префиксным деревом типа Patricia.\\
\pagebreak
\setcounter{subsection}{1}
\subsubsection{Операция поиска строки в префиксном дереве.}
Движемся от корня дерева. Если корень пустой, то поиск неудачный. Иначе, сравниваем ключ в узле с текущей строкой. Для этого воспользуемся функцией, которая вычисляет длину наибольшего общего префикса двух строк заданной длины.\\
В случае поиска нас интересуют три случая: 
 \begin{enumerate}
\item общий префикс может быть пустым, тогда надо рекурсивно продолжить поиск в младшей сестре данного узла, т.е. перейти по ссылке next;
\item общий префикс равен искомой строке x — поиск успешный, узел найден (тут мы существенно используем тот факт, что конец строки за счет наличия в нем терминального символа может быть найден только в листе дерева);
\item общий префикс совпадает с ключом, но не совпадает с x — переходим рекурсивно по ссылке link к старшему дочернему узлу, передавая ему для поиска строку x без найденного префикса.
\end{enumerate}
Если общий префикс есть, но не совпадает с ключом, то поиск также является неудачным.
\subsubsection{Операция вставки новой строки в префиксное дерево.}
Вставка нового ключа (как и в двоичных деревьях поиска) очень похожа на поиск ключа. Естественно с несколькими отличиями. Во-первых, в случае пустого дерева нужно создать узел с заданным ключом и вернуть указатель на этот узел. Во-вторых, если длина общего префикса текущего ключа и текущей строки x больше нуля, но меньше длины ключа (второй случай недачного поиска), то надо разбить текущий узел на два, оставив в родительском узле найденный префикс, и поместив в дочерний узел p оставшуюся часть ключа. После разделения нужно продолжить процесс вставки в узле p строки x без найденного префикса.\\
\newpage
\subsubsection{Удаление ключа из префиксного дерева.}
Как обычно, удаление ключа — самая сложная операция. Хотя в случае префиксного дерева все выглядит не столь страшно. Дело в том, что при удалении ключа удаляется всего один листовой узел, соответствующий суффиксу некоторому удаляемого ключа. Сначала мы находим этот узел, если поиск успешный, то мы его удаляем и возвращаем указатель на младшего брата. \\
В принципе, на этом процесс удаления можно было бы и закончить, однако возникает небольшая проблема — после удаления узла в дереве может образоваться цепочка из двух узлов t и p, в которой у первого узла t имеется единственный дочерний узел p. Следовательно, если мы хотим держать дерево в сжатой форме, то нужно соединить эти два узла в один, произведя операцию слияния. 

\newpage 
\section{Описание программы}
По условию задачи было ясно, что красиво такую программу написать в одном файле не получится и придется её разбить на несколько основных файлов:
\begin{enumerate}
\item main.cpp (содержит основной метод main, LengthString - метод подсчёт длины строки, ToLower - метод перевода строки в нижний регистр)
\item PatriciaTree.h и PatriciaTree.cpp (описание и реализация класса TPatriciaTree)
\item StackContainer.h и StackContainer.cpp (описание и реализация класса TStack и TStackData)
\end{enumerate}
Для удобства сборки проекта были написаны: 
\begin{enumerate}
\item makefile (для сборки и автоматического тестирования проекта из консоли)
\item CMakeLists.txt (для автоматической сборки проекта в CLion, Visual Studio Code, Visual Studio Enterprise 2017)
\end{enumerate}
Также были взяты \enquote{test\_generator.py} и \enquote{wrapper.sh} c GitHub'а[1] и переделаны.
\newpage

\begin{longtable}{|p{8.4cm}|p{7.5cm}|}
\hline
\rowcolor{lightgray}
\multicolumn{2}{|c|} {main.cpp}\\
\hline
void ToLower (char *t, int len)& Перевод char* в нижний регистр.\\
\hline
\hline
unsigned int LengthString (const char *string)& Подсчет строки.\\
\hline
\hline
int main (int argc, char **argv)&Точка входа в программу.\\
\hline
\rowcolor{lightgray}
\multicolumn{2}{|c|} {TPatriciaTree.h и TPatriciaTree.cpp}\\
\hline TPatriciaTree ()&Пустой конструктор.\\ \hline
\hline const unsigned long long *Search (const char *, unsigned int)&Поиск в дереве.\\ \hline
\hline bool Insert (const char *, unsigned int, unsigned long long)&Вставка в дерево.\\ \hline
\hline bool Remove (const char *, unsigned int)&Удаление из дерева.\\ \hline
\hline void Load (const char *)&Загрузка дерева из файла.\\ \hline
\hline void Save (const char *)&Сохранение дерева в файл.\\ \hline
\hline virtual \textasciitilde TPatriciaTree ()&Деструктор.\\ \hline
\hline TPatriciaTree *next&Ссылка на брата.\\ \hline
\hline TPatriciaTree *link&Ссылка на потомка.\\ \hline
\hline char *key&Поле ключа.\\ \hline
\hline unsigned int length&Длина ключа.\\ \hline
\hline unsigned long long data&Поле значения.\\ \hline
\rowcolor{lightgray}
\multicolumn{2}{|c|} {StackContainer.h и StackContainer.hpp}\\
\hline TStackData () &Пустой конструктор.\\ \hline
\hline T \&GetData ()&Получение внутреннего значения.\\ \hline
\hline TStackData <T> *GetNext&Получить следующий элемент стека.\\ \hline
\hline TStackData <T> *GetPrev ()&Получить предыдущий элемент стека.\\ \hline
\hline virtual \textasciitilde TStackData ()&Деструктор элемента стека.\\ \hline
\hline T data& Значение внутри элемента стека.\\ \hline
\hline TStackData <T> *next&Ссылка на следующий элемент стека.\\ \hline
\hline TStackData <T> *prev&Ссылка на предыдущий элемент стека.\\ \hline
\hline TStack ()&Конструктор стека.\\ \hline
\hline T *Top ()&Получения верхнего элемента стека.\\ \hline
\hline void Pop ()&Метод удаление верхнего элемента .\\ \hline
\hline void Push (T \&)& Вставка элемента в стек.\\ \hline
\hline bool Empty ()&Проверка пуст ли стек.\\ \hline
\hline virtual \textasciitilde TStack ()& Деструктор.\\ \hline
\hline TStackData <T> *head& Поле головы стека.\\ \hline
\end{longtable}

\newpage
\section{Код дополнительных файлов}
\lstset{language=BASH}
\begin{itemize}
\item makefile
\begin{lstlisting}
FLAGS=-g -std=c++11 -pedantic -Wall -Werror -Wno-sign-compare -Wno-long-long -O2
CC=g++
LINK=-lm

all: tree main

main: main.cpp
	$(CC) $(FLAGS) -c main.cpp
	$(CC) $(FLAGS) -o Patricia PatriciaTree.o main.o $(LINK)

maind: main.cpp
	$(CC) $(FLAGSD) -c main.cpp
	$(CC) $(FLAGSD) -o Patricia PatriciaTree.o main.o $(LINK)

mainp: main.cpp
	$(CC) $(FLAGS) -c main.cpp
	$(CC) $(FLAGS) -ftest-coverage -fprofile-arcs -o Patricia PatriciaTree.o main.o $(LINK)

tree: PatriciaTree.cpp
	$(CC) $(FLAGS) -c PatriciaTree.cpp

clear:
	rm -f *.o
	rm -fr *.dSYM

runtests:
	rm -rf tests
	mkdir tests
	sh wrapper.sh
\end{lstlisting}
\newpage
\item test\_generator.py
\lstset{language=Python}
\begin{lstlisting}
# -*- coding: utf-8 -*-
import sys
import random
import string
import copy
from random import choice
from string import ascii_uppercase
def get_random_key():
    return ''.join(choice(ascii_uppercase) for i in range(random.randint(1, 256)))

if __name__ == "__main__":
    count_of_tests = 1
    actions = [ "+", "-", "?", "!"]
    acts_file = ["Load test", "Save test"]
    for enum in range( count_of_tests ):
        keys = {}
        save_file = {}
        test_file_name = "tests/{:02d}".format( enum + 1 )
        with open( "{0}.t".format( test_file_name ), 'w' ) as output_file, \
             open( "{0}.a".format( test_file_name ), "w" ) as answer_file:

            for _ in range( random.randint(1000, 1000) ):
                action = random.choice( actions )
                if action == "+":
                    key = get_random_key()
                    value = random.randint( 1, 50)
                    output_file.write("+ {0} {1}\n".format( key, value ))
                    key = key.lower()
                    answer = "Exist"
                    if key not in keys:
                        answer = "OK"
                        keys[key] = value
                    answer_file.write( "{0}\n".format( answer ) )

                elif action == "?":
                    search_exist_element = random.choice( [ True, False ] )
                    key = random.choice( [ key for key in keys.keys() ] ) if search_exist_element and len( keys.keys() ) > 0 else get_random_key()
                    output_file.write( "{0}\n".format( key ) )
                    key = key.lower()
                    if key in keys:
                        answer = "OK: {0}".format( keys[key] )
                    else:
                        answer = "NoSuchWord"
                    answer_file.write( "{0}\n".format( answer ) ) 
                    
                elif action == "-":
                    key = get_random_key()
                    output_file.write("- {0}\n".format(key))
                    key = key.lower()
                    answer = "NoSuchWord"
                    if key in keys:
                        del keys[key]
                        answer = "OK"
                    answer_file.write("{0}\n".format( answer ) )
                
                elif action == "!":
                    act_file = random.choice(acts_file)
                    if act_file == "Save test":
                        output_file.write("{0} {1}\n".format( action, act_file ))
                        save_file = keys.copy()
                        answer = "OK"
                    elif act_file == "Load test":
                        output_file.write("{0} {1}\n".format( action, act_file ))
                        keys = {}
                        keys = save_file.copy()
                        answer = "OK"
                    answer_file.write( "{0}\n".format( answer ) )
                        
\end{lstlisting}
\newpage
\item benchmark.cpp
\lstset{language=C}
\begin{lstlisting}
#include <cstdio>
#include <string.h>
#include <map>
#include <fstream>
#include <iostream>

void LowerString(std::string &);
void Parsing(char &, std::string &, unsigned long long &);


int main() {
    std::ofstream fout("benchTime");
    std::map<std::string, unsigned long long> map;
    std::string buffer;
    char action;
    unsigned long long value;
    clock_t start = clock();
    while(true) {
        Parsing(action, buffer, value);
        if(action == 'E') {
            break;
        }
        switch(action) {
            case '+':
                if(map.count(buffer) == 0) {
                    map[buffer] = value;
                    printf("OK\n");
                } else {
                    printf("Exist\n");
                }
                break;
            case '-':
                if (map[buffer] == 0) {
                    printf("NoSuchWord\n");
                }
                else {
                    map.erase(buffer);
                    printf("OK\n");
                }
                break;
            case 'F':
                if (map[buffer] == 0) {
                    printf("NoSuchWord\n");
                }
                else {
                    printf("OK: %lu\n", map[buffer]);
                }
                break;
        }
    }
    clock_t finish = clock();
    double time = (double) (finish - start)/ CLOCKS_PER_SEC;
    fout << "Std map time:" << time << std::endl;
    fout.close();
    return 0;
}

void LowerString(std::string& buffer) {
    int index = 0;
    while (index < buffer.size()) {
        if (buffer[index] >= 'A' && buffer[index] <= 'Z') {
            buffer[index] += -'A' + 'a';
        }
        index++;
    }
}

void Parsing(char &action, std::string& buffer, unsigned long long &value) {
    char ch;
    size_t i = 0;
    ch = getchar();
    if (ch == EOF) {
        action = 'E';
        return;
    }
    if (ch == '+') {
        action = ch;
        getchar();
        std::cin >> buffer >>value;
        getchar();
        LowerString(buffer);
    } else if (ch == '-') {
        action = ch;
        getchar();
        std::cin >> buffer;
        getchar();
        LowerString(buffer);
    } else if (ch == '!') {
        getchar();
        action = getchar();
        while((ch = getchar()) != ' ') {
            i++;
        }
        scanf("%s", buffer);
        getchar();
    } else {
        action = 'F';
        buffer[i++] = ch;
        while((ch = getchar()) != '\n') {
            buffer[i++] = ch;
        }
        buffer[i] = '\0';
        LowerString(buffer);
    }
}
\end{lstlisting}
\newpage
\item wrapper.sh
\lstset{language=BASH}
\begin{lstlisting}
#!/bin/bash
make
make clear
rm -rf tests
mkdir -p tests
if ! python3 test_generator.py ; then
   echo "ERROR: Failed to python generate tests."
   exit 1
fi

for test_file in `ls tests/*.t`; do
    echo "Execute ${test_file}"
    if ! ./Patricia < $test_file > tmp ; then
        echo "ERROR"
        continue
    fi
    answer_file="${test_file%.*}"

    if ! diff -u "${answer_file}.a" tmp ; then
        echo "Failed"
    else
        echo "OK"
    fi 
done
\end{lstlisting}
\end{itemize}
\newpage
\section{Консоль}
\lstset{language=Bash}
\begin{alltt}
$ make
g++ -g -std=c++11 -pedantic -Wall -Werror -Wno-sign-compare -Wno-long-long -O2 
-c PatriciaTree.cpp
g++ -g -std=c++11 -pedantic -Wall -Werror -Wno-sign-compare -Wno-long-long -O2 
-c main.cpp
g++ -g -std=c++11 -pedantic -Wall -Werror -Wno-sign-compare -Wno-long-long -O2 
-o Patricia PatriciaTree.o main.o -lm
$ ./Patricia
g   
NoSuchWord
+ g 4
OK
+ kij 6
OK
+ de 2
OK
de
OK: 2
- de
OK
+ lp 6
OK
! Save test
OK
- g
OK
- kij
OK
kij
NoSuchWord
! Load test
OK
kij 
OK: 6
- e
NoSuchWord
+ ws 2
OK
\end{alltt}
\newpage
\section{Проверка на утечки памяти }
Тестирование проводилось на файле в 10 000 строк, на системе Arch Linux, с помощью утилиты Valgrind.

\begin{alltt}
$ valgrind ./Patricia < tests/01.t > tmp
==16246== Memcheck, a memory error detector
==16246== Copyright (C) 2002-2017, and GNU GPL'd, by Julian Seward et al.
==16246== Using Valgrind-3.13.0 and LibVEX; rerun with -h for copyright info
==16246== Command: ./Patricia
==16246== 
==16246== 
==16246== HEAP SUMMARY:
==16246==     in use at exit: 0 bytes in 0 blocks
==16246==   total heap usage: 3,362,528 allocs, 3,362,528 frees, 235,824,982 bytes allocated
==16246== 
==16246== All heap blocks were freed -- no leaks are possible
==16246== 
==16246== For counts of detected and suppressed errors, rerun with: -v
==16246== ERROR SUMMARY: 0 errors from 0 contexts (suppressed: 0 from 0)
$ wc tests/01.t
  10000   22476 1016277 tests/01.t
\end{alltt}
\pagebreak

\section{Тест производительности}
Чтобы замерить скорость выполнения программы, в main были добавлены следующие строчки, для замера времени начала и конца выполнения.
\lstset{language=C}
\begin{lstlisting}
...
clock_t start = clock();
...
clock_t end = clock();
    double time = (double)(end-start)/CLOCKS_PER_SEC;
    std::ofstream file("time");
    file << "Time : " << time << std::endl;
    file.close();
\end{lstlisting}
\begin{alltt}
$ ./Patricia < tests/01.t > tmp
$ cat time
Time : 0.004019
$ ./Patricia < tests/02.t > tmp
$ cat time
Time : 0.07181
$ ./Patricia < tests/03.t > tmp
$ cat time
Time : 0.149677
$ wc tests/01.t
  492   954 65297 tests/01.t
$ wc tests/02.t 
  25154   50598 3303285 tests/02.t
$ wc tests/03.t 
  49993   99693 6573363 tests/03.t
$ g++ -g -std=c++11 benchmark.cpp
$ ./a.out < tests/01.t > tmp
$ cat benchTime 
Std map time: 0.008452
$ ./a.out < tests/02.t > tmp
$ cat benchTime 
Std map time: 0.103763
$ ./a.out < tests/03.t > tmp
$ cat benchTime 
Std map time:0.355689
\end{alltt}
\pagebreak

\section{Выводы}
В целом написание дерева не вызвало никаких трудностей и я был приятно удивлен, что Patricia быстрее, чем std::map.\\
Для себя я взялся за написание реализации Patricia и на нескольких других языках, потому что часто имею дело со словарями.
\pagebreak
\begin{thebibliography}{99}
\bibitem{github_t}
{\itshape GitHub одного из семинаристов.} \\URL: \texttt{https://github.com/toshunster/da}. 
\bibitem{wikipedia_sort}
{\itshape Шаблоны функций в С++.} \\URL: \texttt{http://cppstudio.com/post/5165/}. 
\bibitem{wikipedia_sort}
{\itshape Шаблоны функций в С++.} \\URL: \texttt{http://cppstudio.com/post/673/}. 
\bibitem{Kormen}
Томас\,Х.\,Кормен, Чарльз\,И.\,Лейзерсон, Рональд\,Л.\,Ривест, Клиффорд\,Штайн.
{\itshape Алгоритмы: построение и анализ, 2-е издание.} --- Издательский дом \enquote{Вильямс}, 2007. Перевод с английского: И.\,В.\,Красиков, Н.\,А.\,Орехова, В.\,Н.\,Романов. --- 1296 с. (ISBN 5-8459-0857-4 (рус.))
\end{thebibliography}
\end{document}
