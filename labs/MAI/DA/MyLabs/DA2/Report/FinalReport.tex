
%% * Copyright (c) w495, 2009 
%% *
%% * All rights reserved.
%% *
%% * Redistribution and use in source and binary forms, with or without
%% * modification, are permitted provided that the following conditions are met:
%% *     * Redistributions of source code must retain the above copyright
%% *       notice, this list of conditions and the following disclaimer.
%% *     * Redistributions in binary form must reproduce the above copyright
%% *       notice, this list of conditions and the following disclaimer in the
%% *       documentation and/or other materials provided with the distribution.
%% *     * Neither the name of the w495 nor the
%% *       names of its contributors may be used to endorse or promote products
%% *       derived from this software without specific prior written permission.
%% *
%% * THIS SOFTWARE IS PROVIDED BY w495 ''AS IS'' AND ANY
%% * EXPRESS OR IMPLIED WARRANTIES, INCLUDING, BUT NOT LIMITED TO, THE IMPLIED
%% * WARRANTIES OF MERCHANTABILITY AND FITNESS FOR A PARTICULAR PURPOSE ARE
%% * DISCLAIMED. IN NO EVENT SHALL w495 BE LIABLE FOR ANY
%% * DIRECT, INDIRECT, INCIDENTAL, SPECIAL, EXEMPLARY, OR CONSEQUENTIAL DAMAGES
%% * (INCLUDING, BUT NOT LIMITED TO, PROCUREMENT OF SUBSTITUTE GOODS OR SERVICES;
%% * LOSS OF USE, DATA, OR PROFITS; OR BUSINESS INTERRUPTION) HOWEVER CAUSED AND
%% * ON ANY THEORY OF LIABILITY, WHETHER IN CONTRACT, STRICT LIABILITY, OR TORT
%% * (INCLUDING NEGLIGENCE OR OTHERWISE) ARISING IN ANY WAY OUT OF THE USE OF THIS
%% * SOFTWARE, EVEN IF ADVISED OF THE POSSIBILITY OF SUCH DAMAGE.

\documentclass[pdf, unicode, 12pt, a4paper,oneside,fleqn]{article}
\usepackage[utf8]{inputenc}
\usepackage[T2B]{fontenc}
\usepackage[english,russian]{babel}

\frenchspacing

\usepackage{amsmath}
\usepackage{amssymb}
\usepackage{hyperref}
\usepackage{longtable}
\usepackage[table]{xcolor}  
\usepackage{array}
\usepackage{color}
\usepackage{xcolor}

\usepackage{hyperref}


\newcommand{\MYhref}[3][blue]{\href{#2}{\color{#1}{#3}}}%

\usepackage{listings}
\usepackage{alltt}
\usepackage{csquotes}
\DeclareQuoteStyle{russian}
	{\guillemotleft}{\guillemotright}[0.025em]
	{\quotedblbase}{\textquotedblleft}
\ExecuteQuoteOptions{style=russian}

\usepackage{graphicx}

\usepackage{listings}
\lstset{tabsize=2,
	breaklines,
	columns=fullflexible,
	flexiblecolumns,
	numbers=left,
	numberstyle={\footnotesize},
	extendedchars,
	inputencoding=utf8}

\usepackage{longtable}

\def\@xobeysp{ }
\def\verbatim@processline{\hspace{1.2cm}\raggedright\the\verbatim@line\par}

\oddsidemargin=-0.4mm
\textwidth=160mm
\topmargin=4.6mm
\textheight=210mm

\parindent=0pt
\parskip=3pt

\definecolor{lightgray}{gray}{0.9}


\renewcommand{\thesubsection}{\arabic{subsection}}

\lstdefinestyle{customc}{
  belowcaptionskip=1\baselineskip,
  breaklines=true,
  frame=L,
  xleftmargin=\parindent,
  language=C,
  showstringspaces=false,
  basicstyle=\footnotesize\ttfamily,
  keywordstyle=\bfseries\color{green!40!black},
  commentstyle=\itshape\color{gray},
  identifierstyle=\color{black},
  stringstyle=\color{blue},
}

\lstdefinestyle{customasm}{
  belowcaptionskip=1\baselineskip,
  frame=L,
  xleftmargin=\parindent,
  language=[x86masm]Assembler,
  basicstyle=\footnotesize\ttfamily,
  commentstyle=\itshape\color{purple!40!black},
}

\lstset{escapechar=@,style=customc}


\newcommand{\CWPHeader}[1]{\addtocounter{section}{-1}\section{#1}}

% Заголовок курсовой работы
% Единственный аргумент --- ее тема
\newcommand{\CWHeader}[1]{\section*{#1}}

\newcommand{\CWProblem}[1]{\par\textbf{Задача: }#1}

\begin{document}

\begin{center}
\bfseries

{\Large Московский авиационный институт\\ (национальный исследовательский университет)

}

\vspace{48pt}

{\large Факультет информационных технологий и прикладной математики
}

\vspace{36pt}


{\large Кафедра вычислительной математики и~программирования

}


\vspace{48pt}

Лабораторная работа №2 по курсу \enquote{Дискретный анализ}

\end{center}

\vspace{72pt}

\begin{flushright}
\begin{tabular}{rl}
Студент: & М.\,В Спиридонов \\
Преподаватель: & А.\,А. Кухтичев \\
Группа: & М8О-206Б \\
Дата: & \\
Оценка: & \\
Подпись: & \\
\end{tabular}
\end{flushright}

\vfill
\pagestyle{empty}
\begin{center}
\bfseries
Москва, \the\year
\end{center}
\pagebreak

\CWHeader{Лабораторная работа \textnumero 2}

\CWProblem{ 
Необходимо создать программную библиотеку, реализующую указанную структуру данных, на основе которой разработать программу-словарь. В словаре каждому ключу, представляющему из себя регистронезависимую последовательность букв английского алфавита длиной не более 256 символов, поставлен в соответствие некоторый номер, от $0$ до $2^{64} - 1$. Разным словам может быть поставлен в соответствие один и тот же номер.

{\bfseries Вариант дерева:} { PATRICIA.}

{\bfseries Вариант ключа:} {  Строки от 0 до 256 символов, состоящие из символов английского алфавита. }

{\bfseries Вариант значения:} { Числа от $0$ до $2^{64} - 1$}
}
\pagestyle{plain}
\pagebreak

\section{Описание}
Условие задачи подразумевает хранение словаря \enquote{ключ-значение}, в структуре данных типа \enquote{PATRICIA}. Для хранения регистронезависимых строк я отказался от привычного класса \enquote{TString}, реализующего обертку над \enquote{char *}, и класса \enquote{TPair}, реализующего связку \enquote{TString, unsigned long long}, для максимального повышения производительности, т.к. данная вариантом структура данных подразумевает постоянное изменение какой-то части ключа, как при вставке, так и при сериализации.\newline
Суть алгоритма состоит в построении нагруженного дерева. Нагруженное дерево — структура данных реализующая интерфейс ассоциативного массива, то есть позволяющая хранить пары «ключ-значение». В большинстве случаев ключами выступают строки, однако в качестве ключей можно использовать любые типы данных, представимые как последовательность байт.\newline
Нагруженное дерево отличается от обычных n-арных деревьев тем, что в его узлах не хранятся ключи. Вместо них в узлах хранятся части строк и односимвольные метки, а ключем, который соответствует некоему узлу является путь от корня дерева до этого узла, а точнее строка составленная из меток узлов, повстречавшихся на этом пути. В таком случае корень дерева, очевидно, соответствует пустому ключу.\newline
Так как нагруженное дерево реализует интерфейс ассоциативного массива, в нем можно выделить три основные операции, а именно вставку, удаление и поиск ключа. Как и многие деревья, нагруженное дерево обладает свойством самоподобия, то есть любое его поддерево также является полноценным нагруженным деревом. Легко заметить, что все ключи в таком поддереве имеют общий префикс, а значит можно выделить специфичную для этого дерева операцию — получение всех ключей дерева с заданным префиксом за время O(Prefix).\newline
Я реализовал сразу создание \enquote{сжатого}  префиксного дерева. Например, цепочка промежуточных узлов с метками s, m, k заменяется на один узел с меткой smk.
\pagebreak

\section{Описание программы}
\begin{enumerate}
\item main.cpp (содержит основной метод main, ToLower - метод, переводящий char* в нижний регистр, LengthString - метод получения длинны char*)
\item PatriciaTree.h и PatriciaTree.cpp (описание и реализация класса TPatriciaTree)
\item StackContainer.h и StackContainer.hpp (описание и реализация шаблона TStack)
\end{enumerate}
Для удобства сборки проекта были написаны: 
\begin{enumerate}
\item makefile (для сборки и автоматического тестирования проекта из консоли)
\item CMakeLists.txt (для автоматической сборки проекта в CLion, Visual Studio Code, Visual Studio 17)
\end{enumerate}
Также были взяты "test\_generator.py" и "wrapper.sh" c GitHub'а[1] и переделаны.
\newpage

\begin{longtable}{|p{8.4cm}|p{7.5cm}|}
\hline
\rowcolor{lightgray}
\multicolumn{2}{|c|} {main.cpp}\\
\hline
void ToLower (char *t, int len)&Перевод char* в нижний регистр\\
\hline
\hline
unsigned int LengthString (const char *string)&Получение длинны char*\\
\hline
\hline int main ()&main\\
\rowcolor{lightgray}
\multicolumn{2}{|c|} {PatriciaTree.h}\\
\hline
TPatriciaTree ()&Пустой конструктор\\
\hline
\hline
const unsigned long long *Search (const char *, unsigned int)&Поиск по дереву\\
\hline
\hline bool Insert (const char *, unsigned int, unsigned long long)& Вставка в дерево\\
\hline
\hline bool Remove (const char *, unsigned int)& Удаление из дерева\\
\hline void Load (const char *)&Загрузка дерева из бинарного файла\\ \hline
\hline void Save (const char *)&Сохраненине дерева в бинарный файл\\ \hline
\hline virtual \textasciitilde TPatriciaTree ()&Деструктор дерева\\ \hline
\rowcolor{lightgray}
\multicolumn{2}{|c|} {StackContainer.h}\\
\hline template <class T>
class TStackData& Шаблон элемента TStack\\
\hline
\hline  T \&GetData ()& Получение хранимого объекта\\
\hline
\hline TStackData <T> *GetNext ()& Получение следующего элемента\\
\hline
\hline TStackData <T> *GetPrev () & Получение предыдущего элемента\\
\hline
\hline virtual \textasciitilde TStackData ()& Деструктор\\
\hline
\hline template <class T>
class TStack : public TStackData <T>& Шаблон структуры данных \enquote{стек}\\
\hline
\hline  TStack ()& Пустой конструктор\\
\hline  T *Top ()& Получения элемента \enquote{сверху} стека\\
\hline  void Pop ()& Изъятие из стека \\
\hline  void Push (T \&)& Помещение элемента на стек\\
\hline  bool Empty ()& Проверка является ли стек пустым\\
\hline  virtual ~TStack ()& Деструктор\\ \hline
\end{longtable}
\newpage

\section{Код дополнительных файлов}
\begin{itemize}
\item test\_generator.py
\lstset{language=Python}
\begin{lstlisting}
import sys
import random
import string
import copy
from random import choice
from string import ascii_uppercase
def get_random_key():
    return ''.join(choice(ascii_uppercase) for i in range(5))

if __name__ == "__main__":
    count_of_tests = 5
    actions = [ "+", "-", "?", "!"]
    acts_file = ["Load test", "Save test"]
    for enum in range( count_of_tests ):
        keys = {}
        save_file = {}
        test_file_name = "tests/{:02d}".format( enum + 1 )
        with open( "{0}.t".format( test_file_name ), 'w' ) as output_file, \
             open( "{0}.a".format( test_file_name ), "w" ) as answer_file:

            for _ in range( random.randint(10000, 10000) ):
                action = random.choice( actions )
                if action == "+":
                    key = get_random_key()
                    value = random.randint( 1, 10)
                    output_file.write("+ {0} {1}\n".format( key, value ))
                    key = key.lower()
                    answer = "Exist"
                    if key not in keys:
                        answer = "OK"
                        keys[key] = value
                    answer_file.write( "{0}\n".format( answer ) )

                elif action == "?":
                    search_exist_element = random.choice( [ True, False ] )
                    key = random.choice( [ key for key in keys.keys() ] ) if search_exist_element and len( keys.keys() ) > 0 else get_random_key()
                    output_file.write( "{0}\n".format( key ) )
                    key = key.lower()
                    if key in keys:
                        answer = "OK: {0}".format( keys[key] )
                    else:
                        answer = "NoSuchWord"
                    answer_file.write( "{0}\n".format( answer ) ) 
                    
                elif action == "-":
                    key = get_random_key()
                    output_file.write("- {0}\n".format(key))
                    key = key.lower()
                    answer = "NoSuchWord"
                    if key in keys:
                        del keys[key]
                        answer = "OK"
                    answer_file.write("{0}\n".format( answer ) )
                
                elif action == "!":
                    act_file = random.choice(acts_file)
                    if act_file == "Save test":
                        output_file.write("{0} {1}\n".format( action, act_file ))
                        save_file = keys.copy()
                        answer = "OK"
                    elif act_file == "Load test":
                        output_file.write("{0} {1}\n".format( action, act_file ))
                        keys = {}
                        keys = save_file.copy()
                        answer = "OK"
                    answer_file.write( "{0}\n".format( answer ) )
\end{lstlisting}
\lstset{language=BASH}
\item makefile
\begin{lstlisting}
FLAGS=-g -std=c++11 -pedantic -Wall -Werror -Wno-sign-compare -Wno-long-long -O2
CC=g++
LINK=-lm

all: tree main

main: main.cpp
	$(CC) $(FLAGS) -c main.cpp
	$(CC) $(FLAGS) -o Patricia PatriciaTree.o main.o $(LINK)

maind: main.cpp
	$(CC) $(FLAGSD) -c main.cpp
	$(CC) $(FLAGSD) -o Patricia PatriciaTree.o main.o $(LINK)

mainp: main.cpp
	$(CC) $(FLAGS) -c main.cpp
	$(CC) $(FLAGS) -ftest-coverage -fprofile-arcs -o Patricia PatriciaTree.o main.o $(LINK)

tree: PatriciaTree.cpp
	$(CC) $(FLAGS) -c PatriciaTree.cpp

clear:
	rm -f *.o
	rm -fr *.dSYM

runtests:
	rm -rf tests
	mkdir tests
	sh wrapper.sh

\end{lstlisting}
\item benchmark.cpp
\lstset{language=C}
\begin{lstlisting}
#include <cstdio>
#include <map>
#include <fstream>
#include <iostream>

void ToLower (std::string &buffer) {
    int index = 0;
    while (index < buffer.size()) {
        if (buffer[index] >= 'A' && buffer[index] <= 'Z') {
            buffer[index] += -'A' + 'a';
        }
        index++;
    }
}

int main () {
    std::ios::sync_with_stdio(false); //make_faster
    std::ofstream fout("benchTime", std::ios::app);
    std::map <std::string, unsigned long long> map;
    std::string buffer;
    uint64_t data;
    char choice;
    clock_t start = clock();
    if (!std::cin.get(choice)) {
        return 0;
    }
    std::cin.putback(choice);

    while (std::cin >> choice) {
        switch (choice) {
            case '+': {
                std::cin >> buffer >> data;
                ToLower(buffer);

                if (map.count(buffer) == 0) {
                    map[buffer] = data;
                    printf("OK\n");
                } else {
                    printf("Exist\n");
                }
            }
                break;
            case '-': {
                std::cin >> buffer;
                ToLower(buffer);
                if (map[buffer] == 0) {
                    printf("NoSuchWord\n");
                } else {
                    map.erase(buffer);
                    printf("OK\n");
                }
                break;
            }
            default: {
                std::cin.putback(choice);
                std::cin >> buffer;
                ToLower(buffer);
                if (map[buffer] == 0) {
                    std::cout << "NoSuchWord\n" << std::endl;
                } else {
                    std::cout << "OK" << map[buffer] << std::endl;
                }
                break;
            }
        }

    }

    clock_t finish = clock();
    double time = (double) (finish - start) / CLOCKS_PER_SEC;
    fout << "Std map time:" << time << std::endl;
    fout.close();

    return 0;
}
\end{lstlisting}

\end{itemize}
\newpage
\section{Консоль}
\lstset{language=Bash}
\begin{alltt}
$ CounterSort make
g++ -g -std=c++11 -pedantic -Wall -Werror -Wno-sign-compare 
-Wno-long-long -O2 -c TString.cpp
g++ -g -std=c++11 -pedantic -Wall -Werror -Wno-sign-compare 
-Wno-long-long -O2 -c Model.cpp
g++ -g -std=c++11 -pedantic -Wall -Werror -Wno-sign-compare 
-Wno-long-long -O2 -c TVector.cpp
g++ -g -std=c++11 -pedantic -Wall -Werror -Wno-sign-compare 
-Wno-long-long -O2 -c main.cpp
g++ -g -std=c++11 -pedantic -Wall -Werror -Wno-sign-compare 
-Wno-long-long -O2 -o CounterSort TString.o Model.o TVector.o main.o -lm
\end{alltt}
\lstset{language=BASH}
\begin{alltt}
$ CounterSort cat 01.t
60693	nmbfyzfqu
43600	piqyjgkhf
25976	irudfyioj
3679	duqjersbu
38028	wuxtiiogk
3639	mefjaasbe
3915	loxlyyqcs
42023	sslttackw
58339	zoxkqedkb
$ CounterSort ./CounterSort < 01.t > 01.answ
$ CounterSort cat 01.answ
3639	mefjaasbe
3679	duqjersbu
3915	loxlyyqcs
25976	irudfyioj
38028	wuxtiiogk
42023	sslttackw
43600	piqyjgkhf
58339	zoxkqedkb
60693	nmbfyzfqu
$ CounterSort cat 02.t
14517	tyxzddarc
64044	xelwhtsro
39337	kwktcfiiu
9162	qjevarqqs
64981	mrwlghlcs
9480	vagjwzilk
44021	egaeqezha
41017	xpcfbjuly
51997	jyjntrrgf
41622	pkabowgux
44235	yborezsfg
$ CounterSort ./CounterSort < 02.t > 02.answ
$ CounterSort cat 02.answ
9162	qjevarqqs
9480	vagjwzilk
14517	tyxzddarc
39337	kwktcfiiu
41017	xpcfbjuly
41622	pkabowgux
44021	egaeqezha
44235	yborezsfg
51997	jyjntrrgf
64044	xelwhtsro
64981	mrwlghlcs
\end{alltt}
\newpage
\section{Проверка на утечки памяти }
Тестирование проводилось на файле в 100 000 строк, на системе Arch Linux, с помощью утилиты Valgrind.
\begin{alltt}
[Ср окт 04: 05:53] snowden@arch ~/Projects/DA/da1 [master] $ make
g++ -g -std=c++11 -pedantic -Wall -Werror -Wno-sign-compare 
-Wno-long-long -O2 -c TString.cpp
g++ -g -std=c++11 -pedantic -Wall -Werror -Wno-sign-compare 
-Wno-long-long -O2 -c Model.cpp
g++ -g -std=c++11 -pedantic -Wall -Werror -Wno-sign-compare 
-Wno-long-long -O2 -c TVector.cpp
g++ -g -std=c++11 -pedantic -Wall -Werror -Wno-sign-compare 
-Wno-long-long -O2 -c main.cpp
g++ -g -std=c++11 -pedantic -Wall -Werror -Wno-sign-compare 
-Wno-long-long -O2 -o CounterSort TString.o Model.o TVector.o main.o -lm
[Ср окт 04: 05:54] snowden@arch ~/Projects/DA/da1 [master] $ 
valgrind ./CounterSort < tests/01.t > tmp 
==1417== Memcheck, a memory error detector 
==1417== Copyright (C) 2002-2017, and GNU GPL'd, by Julian Seward et al.
==1417== Using Valgrind-3.13.0 and LibVEX; rerun with -h for copyright info
==1417== Command: ./CounterSort
==1417== 
==1417== 
==1417== HEAP SUMMARY:
==1417== in use at exit: 0 bytes in 0 blocks
==1417== total heap usage: 1,517,400 allocs, 1,517,400 frees, 9,380,563 bytes 
allocated
==1417== 
==1417== All heap blocks were freed -- no leaks are possible
==1417== 
==1417== For counts of detected and suppressed errors, rerun with: -v
==1417== ERROR SUMMARY: 0 errors from 0 contexts (suppressed: 0 from 0)
[Ср окт 04: 05:54] snowden@arch ~/Projects/DA/da1 [master] $ wc tests/01.t 
100000 200000 1083130 tests/01.t
\end{alltt}
\pagebreak

\section{Тест производительности}
Чтобы увидеть разницу был созданы файлы размером $10^3$, $10^5$, $10^{6}$, $10*10^{6}$ входных строк и был создан файл бенчмарка, с реализацией сортировки подсчетом и QSort из STD. Исходный код лежит в \enquote{дополнительных файлах} выше.

\begin{alltt}
$ CounterSort make bench
g++ -g -std=c++11 -pedantic -Wall -Werror -Wno-sign-compare -Wno-long-long -O2 
-c benchmark.cpp
g++ -g -std=c++11 -pedantic -Wall -Werror -Wno-sign-compare -Wno-long-long -O2 
-o benchmark benchmark.o -lm
$ CounterSort wc tests/01.t
	1000    2000   15797 tests/01.t
$ CounterSort ./benchmark < tests/01.t
Std SortTime = 0.000080
Counting SortTime = 0.000241

$ CounterSort wc tests/02.t
	100000  200000 1083096 tests/02.t
$ CounterSort ./benchmark < tests/02.t 
Std SortTime = 0.010439
Counting SortTime = 0.005754

$ CounterSort wc tests/03.t
	100000  200000 1083096 tests/03.t
$ CounterSort ./benchmark < tests/03.t 
Std SortTime = 0.095891
Counting SortTime = 0.050415

$ CounterSort wc tests/04.t
	10000000 20000000 108302115 tests/04.t
Std SortTime = 0.984407
Counting SortTime = 0.854098
\end{alltt}
\pagebreak
\section{Дневник отладки}
\begin{enumerate}
\item 29.09.17; 0:07; Сел за написание лабораторной. Написал TString, TPair, TVector. Долго искал информацию по грамотному разделению шаблона на несколько файлов;
\item 1.10.17; 0:49; Первый провал на системе тестирования, убрал rm из makefile;
\item 1.10.17; 1:04; Неверный ответ на первом тесте, программа теряла где-то одно значение. Забыл убрать отладочную строчку с удалением корневого элемента;
\item 1.10.17; 23:24; Превышение временного порога на 10-м тесте. На самом деле я даже не удивился. Изменять размер вектора при каждом добавлении элемента само по себе не очень решение. Исправлял введением доп. переменной в вектор, которая считала сколько всего элементов реально заполнены и в случае её равенства размеру массива, увеличивал размер в 2 раза;
\item 2.10.17; 2:37; Решил дописать кучу никому не нужных методов и оберток, чтобы не нарушать инкапсуляцию;
\item 2.10.17; 4:55; Превышение временного порога на 12-м тесте. Поправил начальный размер вектора с 1-го элемента до 32-х;
\item 2.10.17; 4:59; Программа получила статус \enquote{Ок} на системе автоматического тестирования;
\item 2.10.17; 18:52; Новая версия программы с исправлениями утечек памяти провалила по времени 12-й тест, но это лишь потому что я решил попробовать заслать с начальным массивом в 8 элементов. Также запускал на разных машинках, дописывал код;
\item 2.10.17; 21:41; Решил самоутвердиться и заслал финальную версию программы, которую тестировал пару-тройку часов;
\item 3.10.17; 5:57; Финальная отправка на проверку, перед отправкой отчёта. Внесены мелкие правки по кодстайлу;
\item 3.10.17; 6:27; Засылаю отчёт;
\item 6.10.17; 14:54; Исправил отчёт. Нашёл ошибку в бенчмарке - он показывал не правильное время. Исправил замеры времени, также исправил нарушение инкапсуляции. Пришлось отказаться от шаблона в виде TVector.h+TVector.hpp и написать лишь одну его реализацию, которая зависит только от TPair. Также метод сортировки был перенесён в TVector.cpp.
 \end{enumerate}
 \pagebreak
\section{Выводы}

Сортировка подсчётом применима, если сортируемые числа имеют диапазон возможных значений, который достаточно мал по сравнению с сортируемым множеством. Само написание программы не вызвало особых затруднений. Куда больше сил и времени было потрачено на всевозможные тесты и приведение кода в презентабельный вид.\newline
Весьма много времени ушло на сравнение моей сортировки с другими. Для себя я выяснил, что тестировать все на железе i7 4/8 + SSD (1400мб/1200мб чтение/запись в сек) глупая затея, ибо чтобы увидеть хоть какую-то разницу нужно создавать входные файлы $>10^{18}$ строк. Также написание лабораторной мне показалось чем-то схожим с хакатонами, которые я регулярно посещаю: быстро и красиво написать что-то почти не реально. Написанный за ночь код, нужно доводить еще пару тройку дней до совершенства, проводя уйму времени за тестами.
\newline
На вооружение забрал себе Radix, включающий в себя сортировку подсчётом. Потому что все проведённые мною тесты показали лучшие результаты именно на нем.
\pagebreak
\begin{thebibliography}{99}
\bibitem{github_t}
{\itshape GitHub одного из семинаристов.} \\URL: \texttt{https://github.com/toshunster/da} (дата обращения: 29.09.2017). 
\bibitem{wikipedia_sort}
{\itshape Шаблоны функций в С++.} \\URL: \texttt{http://cppstudio.com/post/5165/} (дата обращения: 30.09.2017). 
\bibitem{wikipedia_sort}
{\itshape Шаблоны функций в С++.} \\URL: \texttt{http://cppstudio.com/post/673/} (дата обращения: 1.10.2017). 
\bibitem{wikipedia_sort}
{\itshape Функция strcpy.} \\URL: \texttt{http://cppstudio.com/post/686/} (дата обращения: 1.10.2017). 
\bibitem{wikipedia_sort}
{\itshape Сортировка подсчётом — Википедия.} \\URL: \texttt{http://ru.wikipedia.org/wiki/Сортировка\_подсчётом} (дата обращения: 16.12.2013). 
\bibitem{Kormen}
Томас\,Х.\,Кормен, Чарльз\,И.\,Лейзерсон, Рональд\,Л.\,Ривест, Клиффорд\,Штайн.
{\itshape Алгоритмы: построение и анализ, 2-е издание.} --- Издательский дом \enquote{Вильямс}, 2007. Перевод с английского: И.\,В.\,Красиков, Н.\,А.\,Орехова, В.\,Н.\,Романов. --- 1296 с. (ISBN 5-8459-0857-4 (рус.))
\bibitem{wikipedia_sort}
{\itshape Bitbucket лектора.} \\URL: \texttt{https://bitbucket.org/nkmakarov/da4students/} (дата обращения: 2.10.2017). \end{thebibliography}
\end{document}
